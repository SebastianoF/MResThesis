
\qquad
\pagestyle{empty}
\newpage



% % % % % % % % % % % % % % % % % % % % % % % % % % % % % % % % % % % % % %
% % SUB SECTION
% % % % % % % % % % % % % % % % % % % % % % % % % % % % % % % % % % % % % %
\section*{Abstract}

%Medical imaging employs techniques and tools belonging to several branches of mathematics and physics that, when applied to morphometric and statistical study of biological shape variability, are collected under the name of \emph{computational anatomy}. 

In medical imaging registration, the study of transformations between patients images is underpinned by the study of geometrical transformations between anatomies. An important set of geometrical transformations that is utilized to model non-rigid deformations is provided by the set of diffeomorphisms (bijective differentiable maps with differentiable inverse).

Comparing two diffeomorphisms, as well as obtaining any meaningful statistics for these elements, is not a straightforward task.

The \emph{log-Euclidean framework}, proposed to tackle this problem, uses some tools from differential geometry; it considers the set of diffeomorphisms with a Lie group structure, having a Lie algebra defined as the tangent space at the origin, as a vector space where to compute statistics.

In this vector space, the operation of composition of diffeomorphisms is not anymore available, but it is possible to define an operation, called \emph{Lie log-composition}, that reflects the composition in the Lie group form the Lie algebra.

Aim of the research here presented is the study of numerical approximations for the computation of the Lie log-composition. 


\vspace{1cm}

\noindent
This document contains 15312 words (counted with detex).\\
$ \text{Chapter} \mapsto \text{Words}:$ 
$1 \mapsto 2304$,
$2 \mapsto 3477 $,
$3 \mapsto 3795 $,
$4 \mapsto 1126 $,
$5 \mapsto 3877$
$6 \mapsto 733$

\qquad
\pagestyle{empty}
\newpage


% % % % % % % % % % % % % % % % % % % % % % % % % % % % % % % % % % % % % %
% % SUB SECTION
% % % % % % % % % % % % % % % % % % % % % % % % % % % % % % % % % % % % % %
\section*{Acknowledgments}

It was not and it is still not easy for me the trail between the study of pure mathematics and the world of medical imaging and biomedical engineering. This route would have never been possible without guides, who have gone before me through the same tortuous paths and who accompanied me in the last year. The main contribution on this side arrived from Marco Lorenzi. With the commitment of a supervisor, he helped me significantly, not only in the development of this thesis, but especially in having introduced me to many of the problems a researcher have to deal with.

I am also grateful to the rowing fellows with whom I shared commitment, challenges and happiness in this first fascinating and demanding year at the GIFT-Surg Project: 
Francois Chadebecq, 
Pankaj Daga, 
Tom Doel, 
George Dwyer,
Michael Ebner,
Luis Herrera,
Ioannis Kourouklides,
Efthymios Maneas,
Sacha Noimark,
Rosalind Pratt,
Marcel Tella,
Gustavo Santos,
Dzhoshkun Shakir,
Guotai Wang and Maria A. Zuluaga.
Fundamental was also the contribution of Jenny Nerny, Rebecca Holmes, Katie Konyn and Liz Zuzikova, that allow students and graduates to focus more on research than paper work.

For the help in the unknown land of infinite dimensional Lie algebra, I have a debt with professor Karl-H. Neeb and dott. Robert Gray.

A non academic, but not less important contribution came on different sides from Andrea Baglione, Gerardo Ballesio, Filippo Ferraris, Valeria Giacosa, Giuliano `er Nuanda' De Rossi, Silvia Porter and Raoul Resta. 

The eclectic buildings of UCL would have been unseen for me without Tom Vercauteren, Sebastien Ourselin and Gary Zhang. Their work and their decision, in a warm day of June 2014, to offer me their support - other than a desk, a laptop and a coffee machine - opened the greatest and most important opportunity I've ever had.
 
In classical music, it is well known that in every concert the two most important tunes are the first one and the last one. Following here the same rule I terminate acknowledging for the great love, effort and patience, Carole Sudre.

% % % % % % % % % % % % % % %


\qquad
\pagestyle{empty}
\newpage



\qquad
\pagestyle{empty}
\newpage
