
\qquad
\pagestyle{empty}
\newpage

% % % % % % % % % % % % % % % % % % % % % % % % % % % % % % % % % % % % % %
% % SUB SECTION
% % % % % % % % % % % % % % % % % % % % % % % % % % % % % % % % % % % % % %
\section*{Acknowledgments}
%
%There are places where problems do not grow on steadfast certainties and in predictable directions.
%Places where there is little room to attempt any construction of solid, sleek and clean mathematical structures, but that demand nevertheless simplicity and precision.

%Here the language created following the abstract, self-contained and rigorous formalisms of pure mathematics is not reliable, exactly as is not reliable the knowledge of the initial condition of any studied phenomenon. Pure math borrowing some of their words and use them as tourists in a foreign land.

It was not and it is still not easy for me the trail between the study of pure mathematics and the world of medical imaging and biomedical engineering. This route would have never been possible without guides, who have gone before me through the same tortuous paths and who accompanied me in the last year. The main contribution on this side arrived from Marco Lorenzi. With the commitment of a supervisor, he helped me significantly, not only in the development of this thesis, but especially in having introduced me to many of the problems a researcher have to deal with.

I am also grateful to the rowing fellows with whom I shared commitment, challenges and happiness in this first fascinating and demanding year at the GIFT-Surg Project: 
Francois Chadebecq, 
Pankaj Daga, 
Tom Doel, 
George Dwyer,
Michael Ebner,
Luis Herrera,
Ioannis Kourouklides,
Efthymios Maneas,
Sacha Noimark,
Rosalind Pratt,
Marcel Tella,
Gustavo Santos,
Dzhoshkun Shakir,
Guotai Wang and Maria A. Zuluaga.

For the help in the unknown land of infinite dimensional Lie algebra, I have a debt with professor Karl-H. Neeb and dott. Robert Gray.

A non academic, but not less important contribution came on different sides from Andrea Baglione, Filippo Ferraris, Gerardo Ballesio, Giuliano `er Nuanda' De Rossi, Silvia Porter and Raoul Resta. 

The eclectic buildings of UCL would have been unseen for me without Tom Vercauteren, Sebastien Ourselin and Gary Zhang. Their work and their decision, in a warm day of June 2014, to offer me their support - other than a desk, a laptop and a coffee machine - opened the greatest and most important opportunity I've ever had.
 
In classical music, it is well known that in every concert the two most important tunes are the first one and the last one. Following here the same rule I terminate acknowledging for the great love, effort and patience, Carole Sudre.

% % % % % % % % % % % % % % %

\qquad
\pagestyle{empty}
\newpage


% % % % % % % % % % % % % % % % % % % % % % % % % % % % % % % % % % % % % %
% % SUB SECTION
% % % % % % % % % % % % % % % % % % % % % % % % % % % % % % % % % % % % % %
\section*{Abstract}

%Medical imaging employs techniques and tools belonging to several branches of mathematics and physics that, when applied to morphometric and statistical study of biological shape variability, are collected under the name of \emph{computational anatomy}. 
Image registration is one of the critical tools in medical imaging. It consists in the process of aligning two or more patient images with the aim of determining and quantifying the occurring anatomical correspondences and differences. 

To model the anatomical variability from one image to the other the set of diffeomorphisms (bijective differentiable maps with differentiable inverse) appears to be an interesting option.

Comparing two diffeomorphisms, as well as obtaining any meaningful statistics for these elements, is not a straightforward task.
Approaching this problem, the \emph{log-Euclidean framework}, proposed to consider the set of diffeomorphisms with a Lie group structure, having a Lie algebra defined as the tangent space at the origin where to compute statistics.

In this local representation the operation of composition of diffeomorphisms is not anymore available. An operation in the tangent space that reflects the composition in the Lie group is therefore required.

Aim of this thesis is to find and compare numerical computations of this operation, called here \emph{log-composition}. 
A fast numerical methods for its computation would improve the computation of \emph{log-demons} registration algorithm.


\begin{flushright}
This document contains 15xxx words. 
\end{flushright}

 





% % % % % % % % % % % % % % % % % % % % % % % % % % % % % % % % % % % % % %
% % SUB SECTION
% % % % % % % % % % % % % % % % % % % % % % % % % % % % % % % % % % % % % %
%\section*{Thesis' Organization}
%%Four chapters, all alike in dignity are presented in this fair thesis: \\
%\begin{enumerate}
%	\item[{\bf Chapter \ref{ch:introduction}}] The first chapter is devoted to the introduction of diffeomorphic image registration and the main recent advancements. Here are introduced advantages and disadvantages of using diffeomorphisms as well as some of the milestone frameworks that led to, or are directly affected by the log-composition.
%	
%	\item[{\bf Chapter \ref{ch:tools}}] The second one is the chapter of mathematical elements and tools. We introduce main definitions and results from differential geometry directly involved in the proposed numerical methods, as Lie logarithm, Lie exponential, connections and parallel transport.
%	The concept of log-composition, around which the research gravitates, is defined as consequence of the need of generalize the BCH formula, and we propose the first steps for its computation. 
%	
%	\item[{\bf Chapter \ref{ch:spatial_transformations}}] In this section we introduce two sets of transformations on which the numerical methods will be tested and compared. The first set consists in the finite dimensional group of rigid body transformation of the plane: here logarithms and exponentials possess a closed form, therefore a ground truth to compare the methods is available.
%	The second set of transformation on which numerical methods are applied are the set of diffeomorphisms corresponding to the infinite dimensional stationary velocity fields in the tangent space. For this second case no closed form are available, but comparing the exponential it is still possible to find a ground truth respect to which compare results.
%	
%	\item[{\bf Chapter \ref{ch:log_computations}}] The algorithm to compute the logarithm (here called \emph{log-algorithm}) presented in \cite{Bossa2007} is grounded on the BCH formula and can be cleanly reformulated using the log-composition. Each numerical method to compute the log-composition become naturally available to challenge its performance.
%  
%	\item[{\bf Chapter \ref{ch:results}}] Here we are at the experimental results. The log-composition applied to rigid and diffeomorphic registration is applied to synthetic data and clinical images. \\
%	
%	
%\end{enumerate}









