
\qquad
\pagestyle{empty}
\newpage


% % % % % % % % % % % % % % % % % % % % % % % % % % % % % % % % % % % % % %
% % SUB SECTION
% % % % % % % % % % % % % % % % % % % % % % % % % % % % % % % % % % % % % %
\section*{Abstract}

Medical imaging employs techniques and tools belonging to several branches of mathematics and physics that, when applied to morphometric and statistical study of biological shape variability, are collected under the name of \emph{computational anatomy}. 
One of its critical tool, image registration, is widely used in both academical studies and applications, and continuously challenges researchers to enhance accuracy, improve reliability and reduce the time of the computations. The use of diffeomorphisms in image registration and the concomitant introduction of the log-euclidean framework to compute statistics, provide an interesting options to model the organs' deformations and to quantify the variations of anatomies.
One of the challenge of this setting are the numerical computation of the Lie exponential of stationary velocity fields (SVF), the Lie logarithm of the corresponding transformation and their combinations as they appear in the BCH formula: concept at the core of the underpinning theory of the \emph{log-demons} algorithm.\\
The necessity of finding fast numerical computation techniques of the BCH formula gave birth to the concept of log-composition presented in this thesis within some strategies for its computation.


% % % % % % % % % % % % % % % % % % % % % % % % % % % % % % % % % % % % % %
% % SUB SECTION
% % % % % % % % % % % % % % % % % % % % % % % % % % % % % % % % % % % % % %
\section*{Thesis' Organization}
%Four chapters, all alike in dignity are presented in this fair thesis: \\
\begin{enumerate}
	\item[{\bf Chapter \ref{ch:introduction}}] The first chapter is devoted to the introduction of diffeomorphic image registration and the main recent advancements. Here are introduced advantages and disadvantages of using diffeomorphisms as well as some of the milestone frameworks that led to, or are directly affected by the log-composition.
	
	\item[{\bf Chapter \ref{ch:tools}}] The second one is the chapter of mathematical elements and tools. We introduce main definitions and results from differential geometry directly involved in the proposed numerical methods, as Lie logarithm, Lie exponential, connections and parallel transport.
	The concept of log-composition, around which the research gravitates, is defined as consequence of the need of generalize the BCH formula, and we propose the first steps for its computation. 
	
	\item[{\bf Chapter \ref{ch:spatial_transformations}:}] In this section we introduce two sets of transformations on which the numerical methods will be tested and compared. The first set consists in the finite dimensional group of rigid body transformation of the plane: here logarithms and exponentials possess a closed form, therefore a ground truth to compare the methods is available.
	The second set of transformation on which numerical methods are applied are the set of diffeomorphisms corresponding to the infinite dimensional stationary velocity fields in the tangent space. For this second case no closed form are available, but comparing the exponential it is still possible to find a ground truth respect to which compare results.
	
	\item[{\bf Chapter \ref{ch:log_computations}:}] The algorithm to compute the logarithm (here called \emph{log-algorithm}) presented in \cite{Bossa2007} is grounded on the BCH formula and can be cleanly reformulated using the log-composition. Each numerical method to compute the log-composition become naturally available to challenge its performance.
  
	\item[{\bf Chapter \ref{ch:results}:}] Here we are at the experimental results. The log-composition applied to rigid and diffeomorphic registration is applied to synthetic data and clinical images. \\
	
	
\end{enumerate}









