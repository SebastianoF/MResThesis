
\qquad
\pagestyle{empty}
\newpage


% % % % % % % % % % % % % % % % % % % % % % % % % % % % % % % % % % % % % %
% % SUB SECTION
% % % % % % % % % % % % % % % % % % % % % % % % % % % % % % % % % % % % % %
\section*{Abstract}

Medical imaging employs techniques and tools belonging to several branches of mathematics and physics that, when applied to morphometric and statistical study of biological shape variability, are collected under the name of \emph{computational anatomy}. 
One of these techniques, image registration, is widely used in both academical studies and applications, and continuously challenges researchers to enhance accuracy, improve reliability and reduce the time of the computations. The use of diffeomorphisms in image registration and the concomitant introduction of the log-euclidean framework to compute statistics, provide an interesting options to model the organs' deformations and to quantify the variations of anatomies.
Main challenges of this setting are the numerical computation of the Lie logarithm and Lie exponential, and their combinations as they appear in the BCH formula: concept at the core of the underpinning theory of the \emph{log-demons} algorithm.
The necessity of finding fast numerical computation techniques of the BCH formula gave birth to the concept of log-composition presented in this thesis, with some strategies for its computation.


% % % % % % % % % % % % % % % % % % % % % % % % % % % % % % % % % % % % % %
% % SUB SECTION
% % % % % % % % % % % % % % % % % % % % % % % % % % % % % % % % % % % % % %
\section*{Thesis' Organization}
%Four chapters, all alike in dignity are presented in this fair thesis: \\
\begin{enumerate}
	\item[{\bf Chapter \ref{ch:introduction}}] The first chapter is devoted to the introduction of diffeomorphic image registration and the main recent advancements. Here are introduced advantages and disadvantages of using diffeomorphisms as well as some of the milestone frameworks that led to, or are directly affected by the log-composition.
	
	\item[{\bf Chapter \ref{ch:tools}}] The second one is the chapter of mathematical elements and tools; called from Lie group theory and hired to service image registration, techniques and tools are formally defined with a particular attention to flows, left translation, push forward, Lie logarithm, Lie exponential, connections and parallel transport. The concept of log-composition, around which the research gravitates, is defined as consequence of the need of generalize the BCH formula. Since its computation is impractical, other paths as Taylor expansion, parallel transport and accelerating convergence series are proposed as feasible alternatives. 
	
	\item[{\bf Chapter \ref{ch:spatial_transformations}:}] Each of the numerical methods for the computation of the log-composition are not computed and tested directly on the group of diffeomorphisms. Initially it they are applied to the finite dimensional group of rigid body transformation, where logarithms and exponentials possess a closed form. This chapter is aimed to present the customization of the tools defined in the previous chapter for the group of rigid body transformation and for the group of diffeomorphisms.
	
	\item[{\bf Chapter \ref{ch:log_computations}:}] Since the algorithm to compute the logarithm (here called \emph{log-computation algorithm}) presented in \cite{Bossa2007} relies on the BCH formula, the log-composition provides a range of methodologies for its computation. Each of the tools presented here for the log-composition become naturally available to challenge its performance.
  
	\item[{\bf Chapter \ref{ch:results}:}] Here we are at the experimental results. The log-composition applied to rigid and diffeomorphic registration is applied to synthetic data and clinical images. \\
	
	
\end{enumerate}









