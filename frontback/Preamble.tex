
\qquad
\pagestyle{empty}
\newpage

% % % % % % % % % % % % % % % % % % % % % % % % % % % % % % % % % % % % % %
% % SUB SECTION
% % % % % % % % % % % % % % % % % % % % % % % % % % % % % % % % % % % % % %
\section*{Acknowledgments}
%
%There are places where problems do not grow on steadfast certainties and in predictable directions.
%Places where there is little room to attempt any construction of solid, sleek and clean mathematical structures, but that demand nevertheless simplicity and precision.
%
%In these places any language created by the formalisms of pure mathematics does not work smoothly and we can not do any better than borrowing some of their words and use them as tourists in a foreign land.

It was not, and it is still not easy for me, the trail from the garden of pure mathematics to the world of biomedical applications. But it is crucial to have guides, who have gone before me through the same tortuous paths and who are accompanying me. The main contribution on this side arrived from Marco Lorenzi. With the commitment of a supervisor, he helped me significantly, not only in the development of this thesis, and in solving various problems and doubts, but especially in having - laboriously - introduced me to the craft of the biomedical research.

I am also grateful to the rowing fellow with whom I shared commitment, challenges and happiness of this first fascinating and demanding year at the GIFT-Surg Project: 
Francois Chadebecq, 
Pankaj Daga, 
Tom Doel, 
George Dwyer,
Michael Ebner,
Luis Herrera,
Ioannis Kourouklides,
Efthymios Maneas,
Sacha Noimark,
Rosalind Pratt,
Marcel Tella,
Gustavo Santos,
Dzhoshkun Shakir,
Guotai Wang and Maria A. Zuluaga.

The eclectic buildings of the UCL would have been unseen for me without Tom Vercauteren, Sebastien Ourselin and Gary Zhang. Their work and their decision, in a warm day of June 2014, to offer me support - other than a desk, a laptop and a coffee machine - at TIG, opened the greatest and important opportunity I've ever had.

 An external, but not less important contribution came from Andrea Baglione, Filippo Ferraris, Gerardo Ballesio, Giuliano `er Nuanda' De Rossi, Silvia Porter and Raoul Resta. 
 
In classic music, it is well known that in every concert the two most important tunes are the first one and the last one. Following here the same rule I terminate acknowledging for her greatest love, support and patient, Carole Sudre.

% % % % % % % % % % % % % % %

\qquad
\pagestyle{empty}
\newpage


% % % % % % % % % % % % % % % % % % % % % % % % % % % % % % % % % % % % % %
% % SUB SECTION
% % % % % % % % % % % % % % % % % % % % % % % % % % % % % % % % % % % % % %
\section*{Abstract}

%Medical imaging employs techniques and tools belonging to several branches of mathematics and physics that, when applied to morphometric and statistical study of biological shape variability, are collected under the name of \emph{computational anatomy}. 
\emph{Image registration} is one of the critical tool in medical imaging. It consists in the process of alignment of two or more patients' images with the aim of determining and quantifying the occurring anatomies' correspondences and differences.
It is widely used in both academical studies and applications, and it continuously challenges researchers to enhance accuracy, improve reliability and reduce computational time.

The introduction of the \emph{Lie group of diffeomorphisms} from differential geometry to image registration provides an interesting set of transformations to model the organs' deformations, and to quantify them, thanks to the \emph{log-Euclidean framework} for the computation of statistics.
This machinery enables the representation of diffeomorphisms embedded in the one-parameter subgroup as \emph{stationary velocity field} (SVF) in the Lie group's tangent space (the \emph{Lie algebra}). Despite the fact that statistics can be computed easily and consistently, one of the challenges remains the numerical computation of the map that transforms SVF in the corresponding diffeomorphisms, the Lie exponential, as well as the numerical computation of its inverse function, the Lie logarithm. 

These two transformations allow in particular the computation of the composition of diffeomorphisms in the tangent space, operation called in this thesis \emph{log-composition}. 
The necessity of finding fast numerical methods for its computation arises for example in the \emph{log-demons} and in the \emph{symmetric log-demon} registration algorithm.

In this research we analyze existing numerical methods for the computation of the log-composition, based on the BCH formula and we compare them with two new methods developed in this research, one based on the Taylor expansion and the other on the geometrical concept of parallel transport. 

\begin{flushright}
This document contains 15000 words. 
\end{flushright}

 





% % % % % % % % % % % % % % % % % % % % % % % % % % % % % % % % % % % % % %
% % SUB SECTION
% % % % % % % % % % % % % % % % % % % % % % % % % % % % % % % % % % % % % %
%\section*{Thesis' Organization}
%%Four chapters, all alike in dignity are presented in this fair thesis: \\
%\begin{enumerate}
%	\item[{\bf Chapter \ref{ch:introduction}}] The first chapter is devoted to the introduction of diffeomorphic image registration and the main recent advancements. Here are introduced advantages and disadvantages of using diffeomorphisms as well as some of the milestone frameworks that led to, or are directly affected by the log-composition.
%	
%	\item[{\bf Chapter \ref{ch:tools}}] The second one is the chapter of mathematical elements and tools. We introduce main definitions and results from differential geometry directly involved in the proposed numerical methods, as Lie logarithm, Lie exponential, connections and parallel transport.
%	The concept of log-composition, around which the research gravitates, is defined as consequence of the need of generalize the BCH formula, and we propose the first steps for its computation. 
%	
%	\item[{\bf Chapter \ref{ch:spatial_transformations}}] In this section we introduce two sets of transformations on which the numerical methods will be tested and compared. The first set consists in the finite dimensional group of rigid body transformation of the plane: here logarithms and exponentials possess a closed form, therefore a ground truth to compare the methods is available.
%	The second set of transformation on which numerical methods are applied are the set of diffeomorphisms corresponding to the infinite dimensional stationary velocity fields in the tangent space. For this second case no closed form are available, but comparing the exponential it is still possible to find a ground truth respect to which compare results.
%	
%	\item[{\bf Chapter \ref{ch:log_computations}}] The algorithm to compute the logarithm (here called \emph{log-algorithm}) presented in \cite{Bossa2007} is grounded on the BCH formula and can be cleanly reformulated using the log-composition. Each numerical method to compute the log-composition become naturally available to challenge its performance.
%  
%	\item[{\bf Chapter \ref{ch:results}}] Here we are at the experimental results. The log-composition applied to rigid and diffeomorphic registration is applied to synthetic data and clinical images. \\
%	
%	
%\end{enumerate}









