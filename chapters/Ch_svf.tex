
\chapter{The Set of Stationary Velocity Fields}\label{ch:svf}

\begin{flushright}
	\emph{Accurate reckoning: the entrance into knowledge of all existing things and all obscure secrets.\\
		- Ahmes, 1800 B.C.}
\end{flushright}

The set of diffeomorphisms can be seen as an infinite dimensional Lie group. For these reasons xxx we reduce the set of transformation to the SVF. This has the following positive consequences xxx and it has been applied in xxx . Nevertheless reducing the set of transformation to the SVF has bring new issues and challenges xxx - limitations - 

% % % % % % % % % % % % % % % % % % % % % % % % % % % % % % % % % % % % % %
% % SECTION
% % % % % % % % % % % % % % % % % % % % % % % % % % % % % % % % % % % % % % 
\section{Set and Set Only}\label{se:svf_set}
Let $\Omega$ be an open connected subset of $\mathbb{R}^d$ containing the origin.  We define $\text{Diff}(\Omega)$ the infinite dimensional Lie group of diffeomorphism over $\Omega$ with neutral element $\text{e}$:
\begin{align*}
\text{Diff}:= \lbrace f:\mathbb{R}^d \longrightarrow \mathbb{R}^d \mid \text{ diffeomorphism } \rbrace
\end{align*}
xxx Short non-formal part about recognizing $\text{Diff}(\Omega)$ as a Lie group. Banach manifold and Frechet manifold. What does imply the naive theory of infinite dimensional manifold.



% % % % % % % % % % % % % % % % % % % % % % % % % % % % % % % % % % % % % %
% % SUBSECTION
% % % % % % % % % % % % % % % % % % % % % % % % % % % % % % % % % % % % % % 
\section{Some Tools for the Infinite Dimensional Case}

It can be proved that the Lie algebra of $\text{Diff}(\mathbb{R}^{d})$ is isomorphic to the Lie algebra of the vector field over $\mathbb{R}^{d}$.
\begin{align}\label{eq:mainaimliealgebra}
Lie(\text{Diff}(\mathbb{R}^{d})) = \mathcal{V}(\mathbb{R}^{d})
\end{align}
To Visualize the meaning of this isomorphism we can consider the following diagram:

\[
\begindc{\commdiag}[40]
%
\obj(0,22)[TR1]{$T_{\mathbf{0}}\mathbb{R}^{D}$}
\obj(30,22)[TR2]{$T_{\rho(\mathbf{0})}\mathbb{R}^{D}$}
%
\obj(0,0)[R1]{$\mathbb{R}^{D}$}
\obj(30,0)[R2]{$\mathbb{R}^{D}$}
\obj(-20,0)[R]{$\mathbb{R}$}
%
\obj(30,-20)[Rlow]{$\mathbb{R}$}


%orizzontali
\mor{TR1}{TR2}{$\rho_{\star}$}
\mor{R1}{R2}{$\rho$}
\mor{R}{R1}{$\gamma$}
%verticali
\mor{TR1}{R1}{}%[1,1]
\mor{TR2}{R2}{}%[1,1]
\mor{R2}{Rlow}{$f$}%[1,1]
%oblique
\mor{R}{TR1}{$\gamma'(0)$}[1,1]
\mor{R}{TR2}{$(\rho\circ\gamma)'(0)$}[2,1]
\mor{R1}{Rlow}{$f\circ \rho$}[1,1]

\enddc
\]

\vspace{0.4cm}

where $(\rho_{\star})_{\mathbf{0}}$ is the push forward of $\rho$, defined as follows: 
\begin{align*}
\rho_{\star} : T_{\mathbf{0}} \mathbb{R}^{D} & \longrightarrow  T_{F(\mathbf{0})}\mathbb{R}^{D}   
\\
\mathbf{v}  &\longmapsto  \rho_{\star} \mathbf{v}  : \mathcal{C}^{\infty}(\mathbb{R}^{D})  \longrightarrow   \mathbb{R} 
\\
& \qquad \qquad \qquad \quad f \longmapsto \rho_{\star} \mathbf{v}f 
:= 
\mathbf{v}(f\circ \rho) 
\end{align*}
We can consider the first floor of the diagram as the group of diffeomorphism of $\mathbb{R}^{d}$ and the second floor of the diagram as the algebra of  the continuous function from $\mathbb{R}^{d}$ to $\mathbb{R}^{d}$. 
For $\rho \in \text{Diff}(\mathbb{R}^{D})$ and $\gamma : (-\eta,\eta) \rightarrow \mathbb{R}^{D}$ such that $\gamma(0) = \mathbf{0}$, then $(\rho\circ \gamma)'(0)$ belongs to $T_{\rho(\mathbf{0})}\mathbb{R}^{D}$ and $(\rho_{\star})$ is a continuous function from $\mathbb{R}^{D}$ to $\mathbb{R}^{D}$ and belongs to $\text{Lie}(\text{Diff}(\mathbb{R}^{D}))$.


\begin{lemma}[existence]\label{le:existencelemma}
	Be $p \in \text{Diff}(\mathbb{R}^{d})$, then exists a $\mathbf{v}$ in the Lie algebra $\mathcal{V}(\mathbb{R}^{d})$, such that 
	\begin{align*}
	\euclideanMetric{p - \exp(V)} < \delta
	\end{align*}
	for some $\delta$ and for some metric in $\text{Diff}(\mathbb{R}^{d})$.
\end{lemma}
\begin{proof}
	Investigate a proof to define $\delta$.
\end{proof}
%
\begin{lemma}[identity lemma]\label{le:idlemma}
	Be $p\in \text{Diff}(\mathbb{R}^{d})$, such that $p= \exp(\mathbf{v} )$ for some $\mathbf{v} \in\mathfrak{g}$. Then
	\begin{align*}
	\exp(-\mathbf{v} )\circ p = p \circ \exp(-\mathbf{v} ) = e
	\end{align*}
\end{lemma}
\begin{proof}
	\begin{align*}
	\exp(-\mathbf{v})\circ p = \exp(-\mathbf{v})\circ \exp(\mathbf{v}) =  \varphi_{1}\varphi_{-1} = \varphi_{0} = e \\
	p \circ \exp(-\mathbf{v}) = \exp(\mathbf{v})\circ \exp(-\mathbf{v}) =  \varphi_{-1}\varphi_{1} = \varphi_{0} = e
	\end{align*}
\end{proof}
%
%\begin{lemma}
%	If $p$ is an element of $\text{Diff}(\mathbb{R}^{d})$ defined as before, then $p$ belongs also to the Lie algebra $\mathfrak{g}$. 
%\end{lemma}
%\begin{proof}
%	As done for the proof of the property \ref{pr:matrixLiealgebra} we can consider the path
%	\begin{align*}
%	\gamma : [0,1] & \longrightarrow  \mathbb{G} \\
%	t &\longmapsto I+tg
%	\end{align*}
%	as the path joining $Id$ and $g\in \mathbb{G}$, it follows
%	\begin{align*}
%	\frac{d }{dt}(Id+tg)~\Bigr|_{t = 0}  = g \in \mathbb{G}
%	\end{align*}
%	TO BE VERIFIED!
%\end{proof}

\begin{prop}\label{le:taylorlemma}
	If $\mathbf{v} $ is close to the origin $\exp(\mathbf{v} )$ can be numerically approximated with:
	\begin{align*}
	\exp(\mathbf{v} ) = e + \mathbf{v} 
	\end{align*}
\end{prop}
\begin{proof}
	xxx !!
\end{proof}



xxx Exp and Log function in the infinite dimension case

xxx we can not express the $\exp$ function using the Taylor expansion in the infinite dimensional Lie group $\text{Diff}(\Omega)$. We define it as an unknown function with some features related to the 1-parameter subgroup structure over $\mathbb{G}$:
% definition of exp an log in the infinite case!
We define $\exp$ function as
\begin{align*}
exp : \mathfrak{g} & \longrightarrow  \mathbb{G}   \\
\mathbf{v} &\longmapsto \exp(\mathbf{v}):= \gamma(1)
\end{align*}
The following properties are satisfied:
\begin{enumerate}
	\item $\exp$ is well defined and surjective (at least near 0).
	\item If $\exp(\mathbf{v}) = \gamma(1)$ then $\exp(t\mathbf{v}) = \gamma(t)$.
	\item It satisfies the one parameter subgroup property.
	\item It satisfies the differential equation
	\begin{align*}
	\frac{d}{dt} \exp(t\mathbf{v})~\Bigr|_{t=0} = \mathbf{v}
	\end{align*}
\end{enumerate}

We observe that $\exp$ respects the one parameter subgroup structure of the Lie group $\mathbb{G}$: stretching the tangent vector $\mathbf{v}$ by a parameter $t$, the same stretch is reflected in $\exp(V)$ along the same integral curve.\\ 
In addition exp respect the 1-parameter subgroup structure:
\begin{align*}
\exp((t+s)\mathbf{v}) = \varphi_{t+s} = \gamma(t+s) \qquad \forall t,s \in \mathbb{R}
\end{align*}
moreover, if two elements $p_1 , p_2$ of $\mathbb{G}$ belongs to integral curve passing in $e$ of the same integral curve defined by $\mathbf{v}$, their log function are a vectors having the same direction:
\begin{align*}
p_1 = \gamma(t_1), ~ p_2 = \gamma(t_2) \Rightarrow \exists \mathbf{v} \in \mathfrak{g}, ~ \lambda \in \mathbb{R}\mid \log(p_1) = \mathbf{v} , ~ \log(p_2) = \lambda \mathbf{v}
\end{align*}

It follows that for a fixed $t \in \mathbb{R}$ and $\gamma(t) = \exp(\mathbf{v} )$ for some $\mathbf{v} $ in $left\mathfrak{X}(\mathbb{G})$, then $\gamma(1) = \exp(\frac{1}{t}\mathbf{v})$.

xxx Issue related to the image of exp for stationary velocity fields in the finite dimensional case. define $\text{Diff}_{s}(\Omega)$ as the subset of $\text{Diff}(\Omega)$ defined by the images of exp from the tangent space to the Lie group.


xxx We define $log$ function as
\begin{align*}
log : \mathbb{G} & \longrightarrow  \mathfrak{g}    \\
g &\longmapsto log(g)
\end{align*}
such that for $p$ in $\mathbb{G}$ we have $\exp(log(p)) = p$ when $log(p)$ is defined.



% % % % % % % % % % % % % % % % % % % % % % % % % % % % % % % % % % % % % %
% % SUBSECTION
% % % % % % % % % % % % % % % % % % % % % % % % % % % % % % % % % % % % % % 
\section{SVF in Practice}

Three main definitions around which the whole theory of diffeomorphic image registration gravitate are introduced in this section. \\
xxx Define here displacement and deformation.\\
% time dependent vector space
xxx the set of time dependent spatial transformation. We can express it as the set of continuous functions from $\Omega$ to $\mathbb{R}^{d}$ depending on a real parameter in $T\subseteq \mathbb{R}$:
\begin{align*}
\mathcal{V}_{T}(\mathbb{R}^{d}) = \mathcal{V}_{T} := \lbrace V : \Omega \times T &\longrightarrow \mathbb{R}^d \mid \text{ continuous } \rbrace 
\end{align*}
its elements are called \emph{time varying velocity field} (TVVF) and can be expressed as
\begin{align*}
V(\mathbf{x},t) = \sum_{i=1}^{d} v_{i}(\mathbf{x},t) \frac{\partial}{\partial x_{i}}~\Bigr|_{\mathbf{x}} 
\qquad 
\quad 
v_{i} \in \mathcal{C}^{\infty} (\Omega \times T)
\end{align*}
In case $V(\mathbf{x},t) = V(\mathbf{x},s)$ for all $s,t$ real, then $V$ is a \emph{stationary velocity field} (SVF), and the set of the stationary velocity field, second item presented in this section, is defined as
\begin{align*}
\mathcal{V}(\mathbb{R}^{d}) = \mathcal{V} := \lbrace V : \Omega &\longrightarrow \mathbb{R}^d \mid \text{ continuous } \rbrace 
\end{align*}
Their elements can be expressed as 
\begin{align*}
V(\mathbf{x}) = V_{\mathbf{x}} = \sum_{i=1}^{d} v_{i}(\mathbf{x}) \frac{\partial}{\partial x_{i}}~\Bigr|_{\mathbf{x}} \qquad \quad v_{i} \in \mathcal{C}^{\infty} (\Omega)
\end{align*}
While $\mathcal{V}$ and $\mathcal{V}_{T}$ are Lie algebra, $\text{Diff}$ is a Lie group with the operation of composition.\\

% particle free to move over the manifold 
If we imagine a particle starting at the point $\mathbf{x}$ of $\Omega \subseteq \mathbb{R}^{d}$ at time $0$, with velocity vector for each instant of time given by $V(\mathbf{x},t)$, then its trajectory $\gamma = \gamma(t)$ is determined by the ODE:
\begin{align*}
\frac{d\gamma}{dt} = V(\mathbf{x},t)
\end{align*}
In case $V(\mathbf{x},t)$ is a stationary velocity field the equation is \emph{stationary} or \emph{autonomous}.
%Another application of the Cauchy theorem for a single fixed vector lead to the following lemma, that will be useful in the consequent sections
%\begin{lemma}\label{le:vector_curves_corresp}
%	Be $\mathbf{v} = (v_{1}, \dots , v_{D})$ vector in $\Omega \subseteq\mathbb{R}^{d}$, then for some $\eta > 0$ exists a curve over $\mathbb{R}^{D}$ of class $\mathcal{C}^{1}(-\eta,\eta)$
%	\begin{align*}
%	\gamma : (-\eta,\eta) &\longrightarrow \mathbb{R}^{D} 
%	\end{align*}
%	such that $\gamma(0) = \mathbf{0}$ and $\dot{\gamma}(0) = \mathbf{v} $
%\end{lemma}
%\begin{proof}
%	Considering the vector $\mathbf{v}$ as the constant function (continuous and satisfying the Lipschitz condition), we can apply the theorem of existence of a solution for the following Cauchy problem
%	\begin{equation*}
%	\begin{cases}
%	\dot{\gamma}(t) = \mathbf{v} \\
%	\gamma(0) = \mathbf{0}
%	\end{cases}
%	\end{equation*}
%	whose result is the curve $\gamma$ we where searching for.
%\end{proof}

\noindent
xxx if $V^{(t)} = V^{(s)}$ for all $s,t$ real, then we call this vector field \emph{stationary velocity field} (SVF), otherwise are called \emph{time varying velocity field} (TVVF).\\
xxx The set of the SVF can be expressed as
\begin{align*}
\text{SVF} := \lbrace \varphi_{t}(e) \mid t \in \mathbb{R}, \dot{\varphi}_{t}(e) = V_{\varphi_{t}(e)},  V \in \mathfrak{V}(\Omega) \rbrace
\end{align*}
(note that in this way $V$ is not an element of the Lie algebra!! We should have said $V \in left\mathfrak{V}(\text{Diff})$).\\
xxxWe know that SVF are geodesics-complete if a norm over $\text{diff}$ is defined, while SVF are not complete. $\varphi_{t}(e)$ do not spans $\text{Diff}$ i.e. for each point of $\text{Diff}$ may not always pass an integral curve of a left-invariant vector field over $\text{Diff}$. We will consider only the element of $\text{Diff}$ of the form $\varphi_{t}(e)$. We assume also that each vector field is complete. (THIS MUST BE INVESTIGATED LATER!)\\
xxxThanks to the Dini theorem we have that a SVF can be considered locally as an element of a local expression of $\text{Diff}$.
Moreover to each spatial transformation vector field corresponds an element of the one parameter subgroup of local transformation over $\mathbb{R}^2$ (not sure...).\\

xxx practical aspects, discretization, structure as they are considered into the practical side!


% %
% \begin{lemma}\label{le:D1equalsD}
%  Given $\varphi_{t}^{\alpha}(e)$, for each $t \in \mathbb{R}$ exists $V^{\beta} \in left\mathfrak{X}(\mathbb{G})$ such that 
%  \begin{align*}
%    \varphi_{t}^{\alpha}(e) = \varphi_{1}^{\beta}(e) 
%  \end{align*}
% \end{lemma}
% %
% Immediate consequence of the previous lemma is:
% \begin{prop}
% Using the above definitions, it follows that
%  \begin{align*}
%   Diff_{1} = \mathbb{G}
%  \end{align*}
% \end{prop} 
% We defined $Diff_{1}$ in order to reflect the lenght of the vectors in $\mathfrak{g}$ immediately over $\mathbb{G}$. This lead to have a sraight correspondence between vectors in the Lie algebra and elements in the Lie group, that we will investigate in the next section.



% % % % % % % % % % % % % % % % % % % % % % % % % % % % % % % % % % % % % %
% % SUBSECTION
% % % % % % % % % % % % % % % % % % % % % % % % % % % % % % % % % % % % % % 
%\subsection{Metrics in $\text{Diff}(\Omega)$}
%Relying on the Lie algebra tangent structure it is possible to define a family of metrics in the Lie group as follows:
%\begin{align*}
%\text{dist}(p,q) := \euclideanMetric{\mathbf{u}  - \mathbf{v} }_{\mathfrak{g}}
%\end{align*}
%$(\text{Diff}(\Omega), \text{dist})$ is a metric space. For the finite dimensional case if $\mathfrak{g}$ is complete then $(Diff, dist)$ is complete (.... investigate the if it is complete even in the infinite dimensional case!).\\
%It is possible to define sum in the Lie group, compatible with the metric using the sum in the Lie algebra and the function exp and log.
%	\begin{align*}
%	\odot : \text{Diff}(\Omega) \times \text{Diff}(\Omega) & \longrightarrow  \text{Diff}(\Omega)    \\
%	(p, q) &\longmapsto \exp(log(p) + log(q))
%	\end{align*}
%	Note that $log(p) + log(q)$ is an element of the Lie algebra$\mathfrak{g}$ while  $\exp(log(p) + log(q))$ is in $\mathbb{G}$. Following properties hold
%
%	\begin{enumerate}
%		%
%		\item $\odot$ is well defined.
%		%
%		\item Closure of $\odot$ under inverse element: if $p \in \mathbb{G}$ then $p^{-1} \in \mathbb{G}$.
%		%
%		\item Distance is inversion invariant:
%		\begin{align*}
%		\text{dist}(p,q) = dist(p^{-1},q^{-1})
%		\end{align*}
%		%
%		\item Distance is invariant under $\odot$:
%		\begin{align*}
%		\text{dist}(p,q) = dist(p \odot r,q \odot r)
%		\end{align*}
%		%
%		\item The distance is not invariant under the composition of group structure of $\mathbb{G}$:
%		\begin{align*}
%		\text{dist}(p,q) \neq dist(p\circ r,q \circ r)
%		\end{align*}
%		%
%	\end{enumerate}
%	%


