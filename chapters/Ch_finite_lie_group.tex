
% % % % % % % % % % % % % % % % % % % % % % % % % % % % % % % % % % % % % %
% % % % % % % % % % % % % % % % % % % % % % % % % % % % % % % % % % % % % %
% % SECTION
% % % % % % % % % % % % % % % % % % % % % % % % % % % % % % % % % % % % % %
% % % % % % % % % % % % % % % % % % % % % % % % % % % % % % % % % % % % % %
\chapter{A Lie Group Structure for the Set of Transformation}\label{ch:finite_lie_group}


\begin{flushright}
	\emph{Every working mathematician knows that if one does not control oneself (best of all by examples), then after some ten pages half of all the signs in formulae will be wrong and twos will find their way from denominators into numerators. \\ -V.I. Arnold}
\end{flushright}

%%introductory definition
We consider every group $\mathbb{G}$ as a group of transformation acting on $\mathbb{R}^{d}$, having in mind the particular case $d=2,3$ for 2-dimensional or 3-dimensional images.
We will focus out attention to transformations defined by matrices or diffeomorphism. Other than group they also have the structure of Lie group: they are considered with a maximal atlas that makes them differentiable manifold, in which the composition of two transformation and the inverse of each transformation are well defined differentiable maps:
\begin{align*}
\mathbb{G} \times \mathbb{G} & \longrightarrow  \mathbb{G}    \\
(x,y) &\longmapsto  x y^{-1}
\end{align*}
 Differential geometry is in general a technique to use the well known calculus features and operators on spaces different from the usual $\mathbb{R}^{n}$. Adding the differentiable structure to a group of transformations gives us new handles to hold them: in particular provides the opportunity to define a tangent space to each point of the group (and so a fiber bundle), a space of vector fields, a set of flows and one parameter subgroup as well as other features that enrich this structure. The abstract idea of vector field over a manifold will be concretized for image registration introducing the concepts of \emph{displacement field}, \emph{deformation field} and \emph{velocity field (stationary or time varying)} that will be there presented. Avoid pedantry is as important as to avoid confusions on notations and definitions, therefore it is necessary to call back a few concepts from differential geometry tailored for rigid-body and diffeomorphic image registration, before getting into the heat of the applications. 

% % % % % % % % % % % % % % % % % % % % % % % % % % % % % % % % % % % % % %
% % SECTION
% % % % % % % % % % % % % % % % % % % % % % % % % % % % % % % % % % % % % % 
\section{Velocity Vector Fields and Flows}

Let $\gamma(t)$ be a (continuous) path over a Lie group $\mathbb{G}$, such that $t \in (-\eta,\eta) \subseteq \mathbb{R}$ and $\gamma(0) = p$. If $(C,\psi)$ is a local chart, neighborhood of $p$, the tangent vector of $\gamma$ at the point $p$ can be expressed as
\begin{align*}
\mathbf{u}= \frac{d}{dt}(\psi\circ\gamma)(t) ~\Bigr|_{t=0}
\end{align*}
For different choice of $\gamma$ passing through $p$, we obtain different tangent vectors.
%%%
\[
\begindc{\commdiag}[50]%[50]
\obj(10,10)[G]{$C \subseteq \mathbb{G}$}
\obj(0,0)[R]{$\mathbb{R}$}
\obj(20,0)[Rn]{$\mathbb{R}^{n}$}

\mor{G}{Rn}{$\psi$}
\mor{R}{G}{$\gamma$}
\mor{R}{Rn}{$\psi \circ \gamma$}

\enddc
\]
% 
It can be proved that the set of all of the tangent vector at the point $p$ defines a vector space: the tangent space at $p$, indicated with $T_{p}\mathbb{G}$. It can be proved that this construction do not depend on the local chart's choice. \\
Taking into account the disjoint union of all of the tangent spaces of $\mathbb{G}$ we obtain the tangent bundle $T\mathbb{G}$; it can be proven that it is, in its turn, a differentiable manifold.\\
%% Definition of vector field
Be $\mathbb{G}$ $n$-dimensional Lie group. A \emph{vector field} over $\mathbb{G}$ is a function that assigns at each point $p$ of $\mathbb{G}$, a tangent vector $V_{p}$ in the tangent space $T_{p}\mathbb{G}$, such that $V_{p}$ is differentiable respect to $p$. \\
If $(C; x_{1}, \dots , x_{n}) = (C,\psi)$ is a local chart of $\mathbb{G}$, neighborhood of $p$, then $V_{p}$ can be expressed locally as:
\begin{align*}
V_{p} 
= 
\sum_{i=1}^{n}v_{i}(p) \frac{\partial}{\partial x_{i}}~\Bigr|_{p} 
%=
%v^{i}(p) \partial_{i}\bigr|_{p} 
& & 
v_{i} \in \mathcal{C}^{\infty}(C)
\end{align*}
Using the Einstein summation convention $V_{p}$ is sometime expressed as $V_{p} =  v^{i}(p) \partial_{i}\bigr|_{p} $.
The smooth functions $v_{i}$ define the vector fields in the base 
$(\frac{\partial}{\partial x_{1}}, \dots ,\frac{\partial}{\partial x_{n}})$. The idea of expressing the elements of the base in terms of differential operator reveals the possibility to consider each vector field as a directional derivative over the algebra of smooth functions defined on the manifold.  \\
%% TODO add here an example of the sphere for example! See do carmo vol 1. Do not let more than 5 pages without a concrete example!!
The set of all vector field over $M$, indicated with $\mathcal{V}(M)$, is a real vector space and a module over $\mathcal{C}^{\infty}(M)$:
\begin{align*}
&(V+W)_{p} = V_{p} + W_{p}  &\qquad \forall V, W \in \mathcal{V}(M) \\
&(aV)_{p} = aV_{p} &\qquad \forall a \in \mathbb{R} \\
&(fV)_{p} = f(p)V_{p}  &\qquad \forall V \in \mathcal{V}(M) \quad \forall f \in \mathcal{C}^{\infty}(M)
\end{align*}
Moreover $\mathcal{V}(M)$ acts over $\mathcal{C}^{\infty}(M)$ as follows
\begin{align*}
\mathcal{V}(M) \times \mathcal{C}^{\infty}(M) & \longrightarrow  \mathcal{C}^{\infty}(M) &   \\
(V,f) &\longmapsto  Vf  : \mathcal{C}^{\infty}(M)  \longrightarrow   \mathbb{R} \\
& \qquad \qquad \qquad \quad p \longmapsto (Vf)(p) = V_{p}f
\end{align*}
In the local chart the real number $V_{p}f$ is given by
\begin{align*}
(Vf)(p) = V_{p}f =  \sum_{j=1}^{n}v_{j}(p) \frac{\partial f}{\partial x_{j}}~\Bigr|_{p} 
\end{align*}
and represents the directional derivative of $f$ along the vector $V_{p} \in T_{p}M$.\\
If $(C_{2};y_{1}, \dots , y_{n})$ is another local chart, $p \in C_{2}$, then the change of coordinates can be expressed as follows:
\begin{align*}
V_{p} = \sum_{j=1}^{n} \Big( \sum_{i=1}^{n}v_{i}(p) \frac{\partial y_{j}}{\partial x_{i}}~\Bigr|_{p}  \Big) \frac{\partial }{\partial y_{j}}~\Bigr|_{p}
\end{align*}
A vector field can be time-dependent if each of its vectors varies smoothly within a parameter $t$, otherwise it is time-independent. In this case a continuous function over the set of times $T$ is defined:
\begin{align*}
\mathbb{R} \supseteq T & \longrightarrow  \mathcal{V}(\mathbb{G}) &   \\
t &\longmapsto  V^{(t)}  : \mathbb{G}  \longrightarrow   T\mathbb{G}\\
&  \qquad \qquad \quad p \longmapsto V^{(t)}_{p}
\end{align*}
where $V^{(t)}_{p} $ has local coordinates
\begin{align*}
V^{(t)}_{p} 
=\sum_{i=1}^{n}v_{i}(p,t) \frac{\partial}{\partial x_{i}}~\Bigr|_{p} 
\qquad  
v_{i} \in \mathcal{C}^{\infty}(C\times T)
\end{align*}
Let $V$ a vector field over a differentiable manifold $\mathbb{G}$, an \emph{integral curve} of $V$ is given by
	\begin{align*}
	c : (a,b) & \longrightarrow  \mathbb{G}  \quad \text{such that} \quad 
	\dot{c}(t) = V_{c(t)} \in T_{c(t)}\mathbb{G} ~\forall t\in (a,b)
	\end{align*}
To get the equations of the integral curves, we consider the local expression
\begin{align*}
V
= 
\sum_{i=1}^{n}v_{i} \frac{\partial}{\partial x_{i}} %~\Bigr|_{p} 
%=
%v^{i} \partial_{i} %\bigr|_{p} 
& & 
v_{i} \in \mathcal{C}^{\infty}(C)
\end{align*}
and the unknown curve in the same local chart
\begin{align*}
c(t) = (c_{1}, c_{2}, \dots , c_{n})
\qquad
\dot{c}(t)  = \sum_{i=1}^{n}   \frac{dc_{i}(t)}{dt}   \frac{\partial}{\partial x_{i}} ~\Bigr|_{c(t)} 
\end{align*}
Imposing the condition $\dot{c}(t) = V_{c(t)} $ we get:
\begin{align*}
\sum_{i=1}^{n}   \frac{dc_{i}(t)}{dt}   \frac{\partial}{\partial x_{i}} ~\Bigr|_{c(t)} 
= 
\sum_{i=1}^{n} v_{i}(c(t)) \frac{\partial}{\partial x_{i}} ~\Bigr|_{c(t)}  
\end{align*}
For a given point of the manifold, and considering the integral curves passing for this point we obtain the initial condition $c(0) = p$ for a Cauchy problem :
\begin{equation}
\begin{cases}
\frac{dc_{i}(t)}{dt}  =v_{i}(c_{1}, t_{2}, \dots , c_{n})  \\
c_{i}(0) = p_{i}
\end{cases}
\end{equation}
Thanks to the Cauchy theorem it has a unique solution $\gamma(t)$. The unique integral curve passing through $p$ when $t=0$ is noted by $ c^{(p)}(t)$. \\
Integral curves can be divided in $2$ classes: the one whose domain can be extended to the whole real line $\mathbb{R}$ (in this case $V$ is called \emph{completely integrable vector field}) and the one whose domain is a strict subset of $\mathbb{R}$.
We reminds that the \emph{flow} of the vector field $V$ is the defined as:
\begin{align*}
\Phi_{V}: S\times \mathbb{G} &\longrightarrow \mathbb{G}   \\
(t,p) &\longmapsto  \Phi_{V}(t,p) = c^{(p)}(t)
\end{align*}
where $S = \mathbb{R} $ or $S\subset \mathbb{R}$ if $V$ is or is not respectively completely integrable. Fixing the point $p$, the flow become simply the integral curve passing through $p$; keep $t$ fixed and letting $p$ varying over the manifold, we get the position of each point on the manifold subject to the vector field $V$ at the time $t$. This last idea gives raise to the \emph{one-parameter subgroup}:
\begin{align*}
\forall p \in M \qquad \Phi_{V}(t,p) = \varphi_{t}  \qquad G = \{ \varphi_{t} ~:~ t \in S\}
\end{align*}
\begin{align*}
G\times G &\longrightarrow G   \\
(\varphi_{t_1} ,\varphi_{t_2} ) &\longmapsto  \varphi_{t_1 + t_2} 
\end{align*}
Despite the name, the fact that $G$ forms a group is less important\footnote{In this context: from a group theory point of view the action of the group $(\mathbb{R},+)$ over the manifold has as its orbits, the set of disjoints integral curves.} than considering the compatibility between a sum on the real line and a product between functions. This property will be largely used when dealing with Lie logarithms.
In general a continuous function
\begin{align*}
f : \mathbb{R}\supseteq (-\eta,\eta) & \longrightarrow \mathbb{G}  \qquad f(0) = p
\end{align*}
satisfies the \emph{one parameter subgroup property} if $f(t+s) = f(t) f(s)$ where the last multiplication is the composition on the group. 

% % % % % % % % % % % % % % % % % % % % % % % % % % % % % % % % % % % % % %
% % SECTION
% % % % % % % % % % % % % % % % % % % % % % % % % % % % % % % % % % % % % % 
\section{Push-forward, Left, Right and Adjoint Translation}

Given two Lie group $\mathbb{G}$ and $\mathbb{H}$ linked by the differentiable map $F:\mathbb{G}\rightarrow \mathbb{H}$, then the \emph{push forward} at the point $p$ is defined as the covariant operator
\begin{align*}
(F_{\star})_{p} : T_{p} \mathbb{G} & \longrightarrow  T_{F(p)}\mathbb{H}   \\
V_{p}  &\longmapsto  (F_{\star} V_{p})  : \mathcal{C}^{\infty}(\mathbb{H})  \longrightarrow   \mathbb{R} \\
& \qquad \qquad \qquad \quad f \longmapsto (F_{\star} V_{p})(f) = V_{p}(f\circ F) 
=
v(p)^{i} \partial_{i}(f\circ F)\bigr|_{p} 
\end{align*}
When the point $p$ is implicit by the context it will be omitted: namely $(F_{\star})_{p} = F_{\star}$. \\

In general the push forward gives the right to the vector field $V$ defined over $\mathbb{G}$ to act as a derivative on another manifold $\mathbb{H}$. 
Push forward is well defined since a vector field is completely determined by its action over $\mathcal{C}^{\infty}(\mathbb{H})$.
It can be proved that it is linear, satisfies the Leibnitz rules,  and $(G\circ F )_{\star} = G_{\star} \circ F_{\star}$; moreover, the push forward of the identity is the identity map between vector spaces, and if $F$ is a diffeomorphism, $F_{\star}$ is an isomorphism of vector spaces. \\
The \emph{pull-back}, is defined on the dual space of $\mathbb{G}$ and $\mathbb{H}$ as the contravariant operator of the push forward\footnote{Push-forward is defined between vector spaces, pull-back between space of functions and $V_{p}(F^{\star} f ) =  (v(p)^{i} \partial_{i}\bigr|_{p} )(f\circ F) = v(p)^{i} \partial_{i}(f\circ F)\bigr|_{p} = V_{p}(f\circ F) $.}:
\begin{align*}
F^{\star} : \mathcal{C}^{\infty}(\mathbb{H}) & \longrightarrow  \mathcal{C}^{\infty}(\mathbb{G})    \\
f  &\longmapsto  F^{\star} f  := f\circ F
\end{align*}
The following diagram relates pull-back and push-forward:
\[
\begindc{\commdiag}[29]
\obj(-30,30)[TM]{$T_{p}\mathbb{G}$}
\obj(0,30)[TN]{$T_{p}\mathbb{H}$}

\obj(-30,10)[M]{$\mathbb{G}$}
\obj(0,10)[N]{$\mathbb{H}$}
\obj(30,10)[R]{$\mathbb{R}$}

\mor{TM}{TN}{$F_{\star}$}

\mor{M}{N}{$F$}
\mor{N}{R}{$f$}

\mor{TM}{M}{}[1, 1]
\mor{TN}{N}{}[1, 1]

\cmor((-30,6)(-29,4)(-25,3)(0,3)(25,3)(29,4)(30,6)) 
\pup(0,0){$F^{\star}f$}


\enddc
\]

\noindent
Here we restrict our attention at the case $\mathbb{G}=\mathbb{H}$, thus $F$ is a map that moves the points of $\mathbb{G}$ smoothly.
Each element $p$ of a Lie group $\mathbb{G}$ defines three maps on $\mathbb{G}$:
\begin{enumerate}
	\item \emph{left-translation}:
		\begin{align*}
		L_{p}: \mathbb{G} & \longrightarrow  \mathbb{G} \\
		q &\longmapsto pq
		\end{align*}
	\item \emph{right-translation}:
			\begin{align*}
			R_{p}: \mathbb{G} & \longrightarrow  \mathbb{G}\\
			q &\longmapsto qp
			\end{align*}
	\item \emph{adjoint map} 
			\begin{align*}
			\text{Ad}_{p}: \mathbb{G} & \longrightarrow  \mathbb{G} \\
			q &\longmapsto pqp^{-1}
			\end{align*}
\end{enumerate}
The push forward for the vector field $V$ at the point $q$ are given by:
\begin{enumerate}
	\item \emph{left-translation}:
	\begin{align*}
	(L_{p})_{\star}V_{q}f 
	=
	V_{q}(f\circ L_{p}) 
	=
	\sum_{i=1}^{n} v_{i}(q) \frac{\partial f \circ L_{p}}{\partial x_{i}} ~\Bigr|_{q} 
	=   
	\sum_{i=1}^{n} v_{i}(q) \frac{\partial f }{\partial x_{i}} ~\Bigr|_{pq} 
	\end{align*}
	\item \emph{right-translation}:
	\begin{align*}
	(R_{p})_{\star}V_{q}f 
	=
	V_{q}(f\circ R_{p}) 
	=
	\sum_{i=1}^{n} v_{i}(q) \frac{\partial f \circ R_{p}}{\partial x_{i}} ~\Bigr|_{q} 
	=   
	\sum_{i=1}^{n} v_{i}(q) \frac{\partial f }{\partial x_{i}} ~\Bigr|_{qp} 
	\end{align*}
	\item \emph{adjoint map} 
	\begin{align*}
	(\text{Ad}_{p})_{\star}V_{q}f 
	=
	V_{q}(f\circ \text{Ad}_{p}) 
	=
	\sum_{i=1}^{n} v_{i}(q) \frac{\partial f \circ \text{Ad}_{p}}{\partial x_{i}} ~\Bigr|_{q} 
	=   
	\sum_{i=1}^{n} v_{i}(q) \frac{\partial f }{\partial x_{i}} ~\Bigr|_{pqp^{-1}} 
	\end{align*}
\end{enumerate}
We note that in each expression the coefficient $v_{i}(q)$ remains the same even if the partial derivative is not applied at the point $q$. Therefore the linear combination of the constant coefficients $v_{i}(q)$ can be considered as a scalar product with the elements of the base applied at the function $f$.
Left and right translation of the vector $\mathbf{u}$ can be expressed as scalar product with the \emph{differential}, equivalent concept as the push forward, that emphasizes the scalar product implied in the definition:
\begin{align*}
(DL_{p})_{q} : T_{q} \mathbb{G}& \longrightarrow  T_{pq}\mathbb{G}    \\
\mathbf{u} &\longmapsto  (DL_{p})_{q} \cdot \mathbf{u} 
\end{align*}
\begin{align*}
(DR_{p})_{q} : T_{q} \mathbb{G} & \longrightarrow  T_{qp}\mathbb{G}    \\
\mathbf{u} &\longmapsto  (DR_{p})_{q} \cdot \mathbf{u}
\end{align*}
where $(DL_{p})_{q}$, $(DR_{p})_{q}$ are properly defined vectors that can be expressed local coordinates as follow
\begin{align*}
(DL_{p})_{q} = \sum_{i=1}^{n}\frac{\partial }{\partial x_{i}} ~\Bigr|_{pq}
\qquad 
(DR_{p})_{q} = \sum_{i=1}^{n} \frac{\partial }{\partial x_{i}} ~\Bigr|_{qp}
\end{align*}
Or equivalently linear operators defined as:
\begin{align*}
(DL_{p})_{q} : \mathcal{C}^{\infty}(M) & \longrightarrow  \mathbb{R}   \\
 f &\longmapsto  (DL_{p})_{q} (f) = \frac{\partial f}{\partial x_{i}} ~\Bigr|_{pq} 
 \end{align*}
 \begin{align*}
(DR_{p})_{q} : \mathcal{C}^{\infty}(M) & \longrightarrow  \mathbb{R} \\
f &\longmapsto  (DR_{p})_{q} (f) = \frac{\partial f}{\partial x_{i}} ~\Bigr|_{qp}
\end{align*}
A change of notation $V_{q} = \mathbf{u}$ makes push-forward and differential strikingly equivalent. This holds also for the generic map F:
\begin{align*}
(DF)_{q} (f)= \sum_{i=1}^{n}\frac{\partial f\circ F}{\partial x_{i}} ~\Bigr|_{q}
\qquad
(DF)_{q} (f)\cdot \mathbf{u}  = \sum_{i=1}^{n} u_{i}\frac{\partial f\circ F}{\partial x_{i}} ~\Bigr|_{q}
\end{align*}
The subscript $q$ in $(DL_{p})_{q}$ can be omitted when the tangent space of $\mathbf{u}$ is clear by the context. 

%------------------ definition of left invariant vector field
A vector field $V$ defined over a manifold is \emph{left-invariant} if it is invariant for each left translation. It means that $(L_{q})_{\star} V_{p} = V_{p}$ for any choice of $p$ and $q$. If we consider all of the possible push forward of the left translation applied to a single tangent vector at the origin $\mathbf{v}$ of $T_{e}\mathbb{M}$ we have a unique left-invariant vector field defined as $\mathbf{v}^{L}$ such that
\begin{align*}
\mathbf{v}^{L}_{q} := (L_{q})_{\star} \mathbf{v} \qquad \forall q \in M
\end{align*}
Vice versa every left-invariant vector field $V$ is uniquely represented by $V_{e}$.
 The set of all of the left-invariant vector fields form a linear subspace of the space of the vector field, indicated with $\text{left}\mathcal{V}(M)$. This can be easily proved by:
\begin{align*}
(L_{g})_{\star} (aV +bW) = a (L_{g})_{\star} V + b (L_{g})_{\star} W
\qquad 
\forall V, W \in \mathcal{V}(\mathbb{G}) 
\quad 
\forall a, b \in \mathbb{R}
\end{align*}
In fact for each $h \in \mathbb{G}$ and for each $f \in \mathcal{C}^{\infty}(\mathbb{G})$ the linearity property holds:
\begin{align*}
(L_{g})_{\star} (aV_{h} +bW_{h})f &= (aV_{h} +bW_{h})(f\circ L_{g} ) \\
&= aV_{h}(f\circ L_{g} ) +bW_{h}(f\circ L_{g} ) \\
&= a (L_{g})_{\star} V_{h}f + b (L_{g})_{\star} W_{h}f  
\end{align*}
The linearity property leads to the definition of the group of homomorphism over $\mathbb{G}$. It is the set of all the Lie group homomorphism from $\mathbb{R}$ to $\mathbb{G}$:
\begin{align*}
Hom(\mathbb{R},\mathbb{G}) = \lbrace \varphi: \mathbb{R} \rightarrow \mathbb{G} \mid \varphi(a+b) = \varphi(a) \circ \varphi(b)\phantom{aa} \forall a,b \in \mathbb{R} \rbrace
\end{align*}
Tangent spaces, flows, one-parameter subgroup and Lie group homomorphisms are bounded together by the following remarkable result, which is a most important precondition for the definition of the Lie group exponential, and so deserve to be written in form of a lemma and formally proved. 
\begin{lemma}
	Let $\mathbb{G}$ be a Lie group. For each $\mathbf{v}$ in the tangent space $T_{e}\mathbb{G}$, exists a unique homomorphism $\gamma_{\mathbf{v}}$ in $Hom(\mathbb{R},\mathbb{G})$ (or equivalently a function satisfying the one-parameter subgroup property) such that 
	\begin{align*}
	\dot{\gamma}_{\mathbf{v}}(0) = \mathbf{v}
	\end{align*}
\end{lemma}
\begin{proof}
	The homomorphism $\gamma_{\mathbf{v}}$ coincides with the integral curve $\Phi$ of the left invariant vector field generated by $\mathbf{v}$ passing through the identity. Its uniqueness is then a consequence of the Cauchy theorem. The same theorem also specifies the existence for a small enough neighbour $(-\eta,\eta)\subset \mathbb{R}$. To extend the solution to the whole $\mathbb{R}$ it is enough to consider that $\gamma_{\mathbf{v}}(t+s) = \gamma_{\mathbf{v}}(t) \gamma_{\mathbf{v}}(s)$ for each $s,t \in (-\eta,\eta)$:
	\begin{align*}
	\gamma_{\mathbf{v}}(t+s) = \Phi(t+s,e) =\Phi(t,\gamma_{\mathbf{v}}(s)) = \gamma_{\mathbf{v}}(t) \gamma_{\mathbf{v}}(s)
	\end{align*}
	%completeness missing, should refer to a subset of the real line instead the whole real line!
\end{proof}
We observe that $\gamma_{\mathbf{v}}$ is exactly the one parameter subgroup of $\mathbf{v}^{L}$ defined above, and then we can write $\gamma_{\mathbf{v}}(t) = \Phi(t,e) = \varphi_{e}(t)$.\\

We conclude this paragraph remembering the definition of the Lie algebra of a Lie group that take into account every feature so far introduced:
\begin{definition}
	Given a Lie group $\mathbb{G}$, its Lie algebra $\mathfrak{g}$ is defined as:
	\begin{enumerate}
		\item The vector space $T_{e}\mathbb{G}$  of all of the tangent vector at the identity (or at any other point of the manifold): $\mathfrak{g} := T_{e}\mathbb{G}$.
		\item The set of the left invariant vector Field over $\mathbb{G}$: $\mathfrak{g} := \text{left}\mathcal{V}(\mathbb{G})$.
		\item The set of all of the flows passing through $e$:  $\mathfrak{g} := \{ \Phi(e,t) : t \in S\subseteq \mathbb{R} \}$.
		\item The set of homomorphism $Hom(\mathbb{R},\mathbb{G}) $.
	\end{enumerate}
\end{definition}
The Lie algebra can be also defined independently from a Lie group as a vector space endowed with Lie bracket (bilinear form, antisymmetric, that satisfies the Jacobi identity). In the finite dimensional case given a Lie algebra $\mathfrak{g}$ it can be proved that exists always a Lie group $\mathbb{G}$ such that $\mathfrak{g}$ is the Lie algebra defined over $\mathbb{G}$. This property (third Lie theorem) do not holds anymore infinite dimensional Lie algebra of diffeomorphisms.

% % % % % % % % % % % % % % % % % % % % % % % % % % % % % % % % % % % % % %
% % SECTION
% % % % % % % % % % % % % % % % % % % % % % % % % % % % % % % % % % % % % % 
\section{Lie Exponential, logarithm and Log-composition}\label{se:lie_exp_log_comp}
Let $\mathbf{v}$ be an element in the tangent space $\mathfrak{g}$ and $V\in\text{left}\mathcal{V}(\mathbb{G})$ the unique vector field defined by $\mathbf{v}$ over a local coordinate system around the origin. Let $\Phi_{V}$ be the flow associated with the vector field and $\gamma(t)$ the unique integral curve of $V$ passing through the identity of the group.  
The \emph{Lie exponential} is defined as
\begin{align*}
\exp :  \mathfrak{g} & \longrightarrow  \mathbb{G}  \\
\mathbf{v} &\longmapsto  \exp(\mathbf{v} ) = \gamma(1) \quad \dot{\gamma}(t) = V_{\gamma(t)}, \gamma(0) = e
\end{align*}
It satisfies the following properties:
\begin{enumerate}
	    \item $\exp(\mathbf{v}) = \Phi_{V}(e,1)$.
	    \item $\exp(t\mathbf{v}) =\gamma(t) = \Phi_{V}(e,t)$.
   	 	\item $\exp(\mathbf{v}) = e$ if $\mathbf{v} = \mathbf{0}$.
   	 	\item $\exp(\mathbf{v})\circ \exp(\mathbf{-v})  = e$
	 	\item The exponential function satisfies the one parameter subgroup property:
	 	\begin{align*}
        \exp((t+s)\mathbf{v}) = \gamma(t+s) = \gamma(t)\circ \gamma(s) = \exp(t\mathbf{v})\exp(s\mathbf{v})
	 	\end{align*}
	 	\item $\exp(\mathbf{v})$ is invertible and $(\exp(\mathbf{v}))^{-1} = \exp(-\mathbf{v})$.
		\item $\exp$ is a diffeomorphism between a neighborhood of $\mathbf{0}$ in $\mathfrak{g}$ to a neighborhood of $\text{Id}$ in $\mathbb{G}$.
\end{enumerate}
The neighborhoods of $\mathbb{G}$ and of $\mathfrak{g}$ such that the last property holds, are called \emph{internal cut locus} of $\mathbb{G}$ and $\mathfrak{g}$ respectively. The \emph{cut locus} is the boundary of the internal cut locus\footnote{Here we define cut locus starting from the exp and log function, and in both domains. Traditionally it is defined only on Riemannian manifolds and using the geodesics (see \cite{do1992riemannian}, p. 267). For Levi Civita connection we have that the definition are coincident.}.\\
When we deal with a matrix Lie group of dimension $n$, we have the following remarkable property:
\begin{enumerate}
	\item for all $\mathbf{v}$ in a matrix Lie algebra $\mathfrak{g}$:
	\begin{align*}
     \exp(\mathbf{v}) = \sum_{k=0}^{\infty} \frac{\mathbf{v}^{k}}{k!}
	\end{align*}
 	\item If $\mathbf{u}$ and $\mathbf{v}$ are commutative then $\exp(\mathbf{u} + \mathbf{v}) = \exp(\mathbf{u})\exp(\mathbf{v})$.
	\item If $\mathbf{c}$ is an invertible matrix then $\exp(\mathbf{c}\mathbf{v}\mathbf{c}^{-1}) = \mathbf{c}\exp(\mathbf{v})\mathbf{c}^{-1}$.
	\item $\det(\exp(\mathbf{v})) = \exp(\text{trace}(\mathbf{v}))$
 	\item For any norm, $\euclideanMetric{\exp(\mathbf{v})} \leq \exp(\euclideanMetric{\mathbf{v}})$.
 	\item  $\exp(\mathbf{u} + \mathbf{v}) =\lim_{m\rightarrow \infty} (\exp(\frac{\mathbf{v}}{m})\exp(\frac{\mathbf{v}}{m}))^{m}$
 	\item If $\exp(\mathbf{w}) = \exp(\mathbf{u}) \circ \exp(\mathbf{v})$ then $\exp(\mathbf{-w}) = \exp(\mathbf{-v}) \circ \exp(\mathbf{-u})$.
 	\item For $\text{Ad}$ adjoint map\footnote{Lie group action defined as
 		 as 
 		 \begin{align*}
 		 \text{Ad} :  \mathbb{G} & \longrightarrow  Aut(\mathbb{G}) &   \\
 		 \mathbf{u} &\longmapsto  \text{Ad}_{\mathbf{u}}  : \mathbb{G}  \longrightarrow  \mathbb{G} \\
 		 & \qquad \qquad \qquad \quad \mathbf{v} \longmapsto \text{Ad}_{\mathbf{u}}\mathbf{v} = \mathbf{u}\mathbf{v}\mathbf{u}^{-1}
 		 \end{align*}
 		 that preserves the lie bracket in the Lie algebra: $\text{Ad}_{\mathbf{u}}[\mathbf{v}, \mathbf{w} ] = [\text{Ad}_{\mathbf{u}} \mathbf{v}, \text{Ad}_{\mathbf{u}} \mathbf{w} ]$   .
 		 } we have $ \exp(\text{Ad}_{\mathbf{u}}\mathbf{v} ) = \text{Ad}_{\mathbf{u}}\exp(\mathbf{v})$
\end{enumerate}
The idea of defining an inverse of the Lie exponential leads to the idea of the Lie logarithm, defined
\begin{align*}
\log : \mathbb{G} & \longrightarrow \mathfrak{g} \\
p &\longmapsto \log (p)  =  \mathbf{v}   
\end{align*}
where $\mathbf{v}  $ is the tangent vector having $p$ as it $\exp$.\\
The idea of the names seems to be justified by the following example:
\begin{example}
If we take the unitary circle in the complex plane $\mathbb{S}^{1}$, and the vertical line $x = 1$ as its tangent at the point $(0,0)$. Each element $\theta$ of the group of rotation of the plane corresponds a point on the circle $\cos(\alpha)+ i\sin(\alpha)$, and so the group of rotation can be identified with the circle. Thanks to the Euler's formula we can write
\begin{align*}
\mathbb{S}^{1} = \{e^{i\alpha}  \mid \alpha \in (-\pi,\pi] \}
\end{align*}
\begin{figure}[!ht]
	\centering
	\includegraphics[scale=0.35]{figures/circle_example.png}
	\caption{Lie algebra of the Lie group of plane's rotation.}
	\label{fig:evolution}
\end{figure}  
The Lie algebra of this group of rotations is the tangent line to the circle at the neutral element $\alpha = 0$, and it is isomorphic to $\mathbb{R}$. Lie logarithm and Lie exponential for this particular case corresponds exactly with the usual logarithm and exponential:
\begin{align*}
\log : \mathbb{S}^{1} & \longrightarrow T_{0} \mathbb{S}^{1} 
\qquad \qquad \quad \quad 
\exp : T_{0} \mathbb{S}^{1}  \longrightarrow \mathbb{S}^{1}
\\
\exp{(i\alpha)} &\longmapsto \log(\exp{(i\alpha)})  =  i\alpha 
\qquad \qquad \quad   
i\alpha \longmapsto \exp(i\alpha)  
\end{align*}
The internal cut locus of the lie group is $(-\pi, \pi)$.
\end{example}

\noindent
If $\mathbb{G}$ is a matrix Lie group of dimension $n$, the following properties hold:
\begin{enumerate}
	\item for all $\mathbf{v}$ in the matrix Lie algebra $\mathfrak{g}$:
	\begin{align*}
	\log(\mathbf{v}) = \sum_{k=1}^{\infty}(-1)^{k+1} \frac{(\mathbf{v}-I)^{k} }{k!}
	\end{align*}
	where $I$ is the identity matrix.
	\item For any norm, and for any $n\times n$ matrix $\mathbf{c}$, exists an $\alpha$ such that 
	\begin{align*}
	\euclideanMetric{ \log(I + \mathbf{c}) - \mathbf{c} }  \leq \alpha \euclideanMetric{ \mathbf{c}}^{2}
	\end{align*}
	\item For any $n\times n$ matrix $\mathbf{c}$ and for any sequence of matrix $\{\mathbf{d}_{j}\}$ such that  $\euclideanMetric{ \mathbf{d}_{j}} \leq \alpha/j^2$ it follows:
		\begin{align*}
		\lim_{k\rightarrow \infty} \big( I + \frac{\mathbf{c}}{k} + \mathbf{d}_{k} \big)^{k} = \exp{(\mathbf{c})}
		\end{align*}
\end{enumerate} 
Here we may see the beginning of the problem we have to deal with for the rest of the research, when passing from the finite dimensional case to the infinite dimensional case.\\
The domain of the logarithm is the matrix Lie group in which only the composition is defined. Nevertheless it is possible to compute $I + \mathbf{c}$, and this still make sense (and satisfy remarkable properties) when applied to the $\log$. On the other side the domain of the exponential is the matrix Lie algebra, but the exponential can be nevertheless applied to a generic matrix.\\
This can be done thanks to the fact that for matrices, $\mathfrak{g}$ and $\mathbb{G}$ are subset of a bigger algebra, the algebra of invertible matrix: in this context operation of sum is still defined over the group that admit only compositions. The sum between element of a group can be performed on a Lie group, every time he and its Lie algebra are subset of a bigger algebra (Kirillov). In these cases infinite series are doors to passes from the structure of group and the algebra. When presenting the rigid body transformation in chapter \ref{ch:rigid_body_transformations} we will see a second couple of access doors based on numerical approximations.

\subsection{Definition of Lie Log-Composition}

We define the Lie Log-composition (Lie to distinguish it from the Affine Log-composition of the next chapter) as inner binary operation on the Lie algebra that reflects the composition on the lie group:
\begin{align*}
\star : \mathfrak{g} \times \mathfrak{g} & \longrightarrow \mathfrak{g}    \\
(\mathbf{v}_{1}, \mathbf{v}_{2}) &\longmapsto \mathbf{v}_{1}\star \mathbf{v}_{2} =  \log(\exp(\mathbf{v}_1)\circ \exp(\mathbf{v}_2))
\end{align*}

 \begin{figure}[!ht]
 	\centering
 	\includegraphics[scale=0.35]{figures/log_composition.png}
 	\caption{graphical visualization of the Lie log-composition.}
 	\label{fig:composition}
 \end{figure}

\noindent
Following properties holds for the Lie log-composition:
\begin{enumerate}
\item $\mathfrak{g} $ with the Lie log-composition $\star$ is a local topological non-commutative group (local group for short): if $C_{\mathfrak{g}}$ is the internal cut locus of $\mathfrak{g}$ then:
\begin{enumerate}
	\item $(\mathbf{u}_{1}\star\mathbf{u}_{2}) \star \mathbf{u}_{3}
	= \mathbf{u}_{1}\star(\mathbf{u}_{2} \star \mathbf{u}_{3})$ for all $\mathbf{u}_{1}, \mathbf{u}_{2}, \mathbf{u}_{3}$ in $C_{\mathfrak{g}}$.
	\item $\mathbf{u}\star\mathbf{0}  = \mathbf{0}\star\mathbf{u} = \mathbf{u}$ for all $\mathbf{u}$ in $C_{\mathfrak{g}}$.
	\item $\mathbf{u}\star(-\mathbf{u} ) = \mathbf{0}$ for all $\mathbf{u}$ in $C_{\mathfrak{g}}$.
\end{enumerate}
\item For all $t,s$ real, such that $(t+s)\mathbf{v}$ is in $C_{\mathfrak{g}}$,
\begin{align*}
(t\mathbf{v})\star (s\mathbf{v}) = (t+s)\mathbf{v}
\end{align*}
And in particular, if the Lie algebra $\mathfrak{g}$ has dimension $1$ the local group structure is compatible with the additive group of the vector space $\mathfrak{g}$.
\item For all $\mathbf{u}$ and $\mathbf{v}$: 

\noindent
xxx property involving metric and log-composition... may results in something interesting for computing statistics.
\begin{align*}
\euclideanMetric{\mathbf{u}\star \mathbf{v}} = ?( \euclideanMetric{\mathbf{u}} , \euclideanMetric{\mathbf{v}})
\end{align*}
\end{enumerate}

% % % % % % % % % % % % % % % % % % % % % % % % % % % % % % % % % % % % % %
% % SUBSECTION
% % % % % % % % % % % % % % % % % % % % % % % % % % % % % % % % % % % % % % 
\section{BCH formula for the Computation of Log-composition}\label{se:bch_formula}

To compute the Lie Log composition, literature provides the BCH formula, defined as the solution of the equation $exp(\mathbf{w}) = \exp(\mathbf{u}) \circ \exp(\mathbf{v})$, for $\bf{u}$ and $\bf{v}$ in the Lie algebra $\mathfrak{g}$:
\begin{align*}
BCH(\mathbf{u},\mathbf{v}) 
= 
\mathbf{u} + \mathbf{v} + \frac{1}{2}[\mathbf{u},\mathbf{v}] + \frac{1}{12}([\mathbf{u},[\mathbf{u},\mathbf{v}]]
+ [\mathbf{v},[\mathbf{v},\mathbf{u}]]) - \frac{1}{24}[\mathbf{v},[\mathbf{u},[\mathbf{u},\mathbf{v}]]] +... 
\end{align*}

\noindent
xxx derivation of the bch formula, constraints on the Lie algebra elements involved in its computation.

% % % % % % % % % % % % % % % % % % % % % % % % % % % % % % % % % % % % % %
% % SUBSECTION
% % % % % % % % % % % % % % % % % % % % % % % % % % % % % % % % % % % % % % 
\section{Taylor Expansion for the Computation of Log-composition}



Once \emph{adjoint action} of $\mathbf{u}$ on the Lia algebra is defined, nested Lie bracket can be reformulated as multiple composition of this operator:
\begin{align*}
	ad_{\mathbf{u}} : \mathfrak{g}  & \longrightarrow \mathfrak{g}  
	\\
	\mathbf{v} &\longmapsto ad_{\mathbf{u}}   =  [\mathbf{u}, \mathbf{v}]
\end{align*}
So
\begin{align*}
	[  \underbrace{   \mathbf{u},[\mathbf{u},... [\mathbf{u}}_{\text{n-times}},\mathbf{v}]...]] =  ad_{\mathbf{u}}^{n}(\mathbf{v})
\end{align*}
In the appendix of xxx Klarsfeld xxx adjoint action are used to provide an expansion of the BCH formula. This can be rewritten as
\begin{align*}
	\mathbf{u}\star \mathbf{v}  = \mathbf{u} + \frac{ ad_{\mathbf{u}} \exp(ad_{\mathbf{u}}) }{ \exp(ad_{\mathbf{u}}) - 1 }  \mathbf{v} + O({\mathbf{v}}^2)
\end{align*}

\noindent
xxx intermediate passages to be written from zachos, blane!

The functional applied to $\mathbf{v}$ can be rewritten as
\begin{align*}
	\frac{ ad_{\mathbf{u}} \exp(ad_{\mathbf{u}}) }{ \exp(ad_{\mathbf{u}}) - 1 }  = \sum_{n=0}^{\infty} \frac{B_{n}}{n!} ad_{\mathbf{u}}^{n} 
\end{align*}
Where $\lbrace B_{n} \rbrace $ is the sequence of the second-kind Bernoulli number\footnote{If first-kind Bernoulli number is used then each term of the summation must be multiplied for $(-1)^{n}$, as did for example in ....Klarsfeld.}.

