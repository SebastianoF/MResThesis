\chapter{Log-composition to Compute the Lie Logarithm}\label{ch:log_algorithm}

%\begin{flushright}
%	I think you might do something better with the time \\ than wasting it in asking riddles that have no answers. \\ -\emph{Alice in Wonderland.}
%\end{flushright}
%
%\vspace{0.5 cm}

\noindent
The \emph{logarithm computation problem} can be stated as follows:
\begin{center}
\emph{
	Given $p$ in a Lie group $\mathbb{G}$, \\ 
	what is the element $\mathbf{u}$ in its Lie algebra $\mathfrak{g}$ \\
	such that $\exp(\mathbf{u}) = p$ ?  
}
\end{center}
There are several numerical methods to compute the approximation of the problem's solution. Arsigny, who first pointed the applications of the Lie logarithm in medical image registration in \cite{Arsigny:MRM:06} and \cite{arsigny2006bi}, proposed the Inverse scaling and squaring. In this chapter we investigate other numerical iterative algorithms for the computation of the Lie logarithm, called here \emph{logarithm computation algorithm}. They consists in a modification of the algorithm presented in \cite{Bossa:08} that is based on the BCH formula, and therefore can be reformulated using the Lie log-composition. 
It turns out that each of the numerical method to compute the Lie log-composition becomes naturally a numerical method for the logarithm computation.\\
The first step toward this direction is to introduce the space of the approximations of a Lie algebra and a the Lie group.


% % % % % % % % % % % % % % % % % % % % % % % % % % % % % % % % % % % % % %
% % 
% % % % % % % % % % % % % % % % % % % % % % % % % % % % % % % % % % % % % % 
\section{Spaces of Approximations}\label{se:space_of_approximation}

\noindent
As seen in section \ref{se:rigid_body_transformations} and \ref{se:svf}, if the element $\mathbf{u}$ of $\mathfrak{se}(2)$ or SVF is small enough we can approximate $\exp(\mathbf{u})$ with $e + \mathbf{u}$. Based on this, we define two approximating functions:
\begin{align*}
\text{app} : \mathfrak{g} & \longrightarrow  \mathfrak{g}^{\sim}    \\
\mathbf{u} &\longmapsto \exp(\mathbf{u}) - e
\end{align*}
\begin{align*}
\text{App} : \mathbb{G} & \longrightarrow  \mathbb{G}^{\sim}   \\
\exp(\mathbf{u}) &\longmapsto e + \mathbf{u}
\end{align*}
Where $\mathfrak{g} ^{\sim}$ is the space of approximations of elements of $\mathfrak{g} $, and $\mathbb{G}^{\sim} $ is the space of approximations of elements in $\mathbb{G}$, defined as
\begin{align*}
\mathfrak{g} ^{\sim} & := \{ \exp(\mathbf{u}) - e \mid \mathbf{u}\in \mathfrak{g}\} \cup \mathfrak{g} \\
\mathbb{G}^{\sim}  & := \{ e + \mathbf{u} \mid \mathbf{u}\in \mathbb{G}\} \cup \mathbb{G} \\
\end{align*}
In general $\mathfrak{g}^{\sim} \neq \mathfrak{g}$ and $\mathbb{G}^{\sim} \neq \mathbb{G}$, but in the considered cases of $\mathfrak{se}(2)$ and SVF, when $\mathbf{u}$ is \emph{small enough}
it follows that $\exp(\mathbf{u}) - e \in \mathfrak{g} $ and $e + \mathbf{u}\in \mathbb{G}$. Therefore the elements of $\mathfrak{g}^{\sim} $ are compatible with all of the operations of the Lie algebra $\mathfrak{g}$ and the elements of $\mathbb{G}^{\sim}$ are compatible with all of the operations of the Lie group $\mathbb{G}$.\\
Lets examine what does \emph{small enough} means in these two cases:
\begin{enumerate}
	\item[$\mathfrak{se}(2)$ -] Since $\mathfrak{se}(2)$ and $SE(2)$ are subset of the bigger algebra $SE(2)$ then exp and log can be defined as infinite series. From 
	\begin{align*}
	\exp(\mathbf{u}) = I + \mathbf{u} + O(\mathbf{u}^2) 
	\end{align*}
	It follows that $\text{app}(\mathbf{u}) - \mathbf{u} \in O(\mathbf{u}^2)$. Thus for all $\mathbf{u}$ in the Lie algebra smaller than $\delta$ for any norm, exists $M(\delta)$ such that
	\begin{align*}
	\euclideanMetric{\text{app}(\mathbf{u}) - \mathbf{u} } < M(\delta) \euclideanMetric{\mathbf{u}^2}
	\end{align*}
	\item[SVF -] In case of SVF we do not have any Taylor series and big-O notation available but, according to the proposition 8.6 at page 163 of \cite{younes2010shapes}, if $\mathbf{u}$ is, for any norm, smaller than $\epsilon<1/C$, where $C$ is the Lipschitz constant in the same norm, then $1 + \mathbf{u}$ is a diffeomorphism. With this condition holds that
	$\text{SVF}^{\sim} = \text{SVF}$. 
\end{enumerate}

\noindent
Therefore, for each small enough $\mathbf{u}$ in $\mathfrak{se}(2)$ or SVF, 
and for the definition of the Lie log-composition (equation \ref{eq:main_def_log_composition}) 
the following properties holds:
\begin{enumerate}
	\item The approximations $\mathbf{u} \simeq   \text{app} (\mathbf{u})$, $\exp(\mathbf{u}) \simeq   \text{App} (\exp(\mathbf{u})) $ are bounded.
	\item $\mathbf{u} = \mathbf{v} \oplus (-\mathbf{v} \oplus  \mathbf{u} )$
	\item $\text{app} (\mathbf{v} \oplus  \mathbf{u}) = \exp(\mathbf{v})\circ\exp(\mathbf{u}) - 1 \in \mathfrak{g} ^{\sim}$
\end{enumerate}

\noindent
With this machinery, we can reformulate the algorithm presented in \cite{Bossa:08} for the numerical computation of the Lie logarithm using the Lie log-composition.

% % % % % % % % % % % % % % % % % % % % % % % % % % % % % % % % % % % % % %
% % 
% % % % % % % % % % % % % % % % % % % % % % % % % % % % % % % % % % % % % % 
\section{The Logarithm Computation Algorithm using Lie Log-composition}

If the goal is to find $\mathbf{u}$ when its exponential is known, we can consider the sequence transformations $\{\mathbf{u}_{j}  \}_{j=0}^{\infty}$ that approximate $\mathbf{u}$ as consequence of
\begin{align*}
\mathbf{u} = \mathbf{u}_{j} \oplus  (-\mathbf{u}_{j}  \oplus  \mathbf{u} ) \Longrightarrow
\mathbf{u} \simeq \mathbf{u}_{j} \oplus  \text{app}(-\mathbf{u}_{j}  \oplus  \mathbf{u} )
\end{align*}
This suggest that a reasonable approximation for the $(j+1)$-th element of the series can be defined by
\begin{align*}
\mathbf{u}_{j+1} & :=  \mathbf{u}_{j} \oplus  \text{app}(-\mathbf{u}_{j}  \oplus  \mathbf{u} )
\end{align*}
If we chose the initial value $\mathbf{u}_{0}$ to be zero, then the algorithm presented in \cite{Bossa:08}  becomes:
%\begin{align*}
%p= \exp(\mathbf{v}) &= (\exp(\mathbf{v})\circ \exp(-\mathbf{v}))\circ \exp(\mathbf{v})\\
%&= \exp(\mathbf{v})\circ (\exp(-\mathbf{v})\circ p)\\
%&= \exp(\mathbf{v})\circ \exp(\delta \mathbf{v})\\
%&\approx \exp(\mathbf{v})\circ \exp(\tilde{\delta} \mathbf{v})
%\end{align*}
\begin{equation}\label{eq:bossa_reformulated}
\begin{cases}
\mathbf{u}_0 = 0 \\
\mathbf{u}_{j+1} = \mathbf{u}_{j} \oplus  \text{app}(-\mathbf{u}_{j}  \oplus  \mathbf{u} )
\end{cases}
\end{equation}
Making explicit the Lie log-computation and the approximation involved, it follows:
\begin{align}
\mathbf{u}_{j+1} 
&=
\mathbf{u}_{j} \oplus \big( \exp(-\mathbf{u}_{j})  \circ   \exp(\mathbf{u}) - e \big)\\
&=
 \log\Big( \exp( \mathbf{u}_{j}) \circ \exp \big( \exp(-\mathbf{u}_{j})  \circ  \varphi - e \big) \Big)
\end{align}
where $\exp(\mathbf{u}) = \varphi$ is given by the problem, and $\mathbf{u}_{j}$ by the previous step. The BCH provides the analytic solution of the second member, while 
numerical methods to compute the Lie log-composition become numerical methods to compute the Lie logarithm.

% % % % % % % % % % % % % % % % % % % % % % % % % % % % % % % % % % % % % %
% % 
% % % % % % % % % % % % % % % % % % % % % % % % % % % % % % % % % % % % % % 
\subsection{Truncated BCH Strategy}
At each step, we compute the approximation $\mathbf{v}_{j+1}$ with the $k$-th truncation of the BCH formula. The compact form of the algorithm is given by:
\begin{equation}\label{eq:bossa_bch_strat}
\begin{cases}
\mathbf{u}_0 = 0 \\
\mathbf{u}_{j+1} = \text{BCH}^{k}(\mathbf{u}_{j}, \text{app}(-\mathbf{u}_{j}  \oplus  \mathbf{u} ))
\end{cases}
\end{equation}

For $k = 0$, the approximation $\mathbf{u}_{j+1}$ is the sum of the two vectors $\mathbf{u}_{j}$ and $ \text{app}(-\mathbf{u}_{j}  \oplus  \mathbf{u} )$:
\begin{align*}
\text{BCH}^{0}(\mathbf{u}_{j}, \text{app}(-\mathbf{u}_{j}  \oplus  \mathbf{u} ))
&=
\mathbf{u}_{j} + \text{app}(-\mathbf{u}_{j}  \oplus  \mathbf{u} )\\
&=
\mathbf{u}_{j} + \exp(-\mathbf{u}_{j})\circ \varphi  - e 
\end{align*}
When $k=1$, it results 
\begin{align*}
\text{BCH}^{1}(\mathbf{u}_{j}, \text{app}(-\mathbf{u}_{j}  \oplus  \mathbf{u} ))
&=
\mathbf{u}_{j} +  \text{app}(-\mathbf{u}_{j}  \oplus  \mathbf{u} ) + \frac{1}{2}[\mathbf{u}_{j},  \text{app}(-\mathbf{u}_{j}  \oplus  \mathbf{u} )]\\
&=
\mathbf{u}_{j} + \exp(-\mathbf{u}_{j})\circ \varphi  - e + \\
&+ \frac{1}{2}(  \mathbf{u}_{j}\cdot( \exp(-\mathbf{u}_{j})\circ \varphi  - e) -  ( \exp(-\mathbf{u}_{j})\circ \varphi - e)\cdot\mathbf{u}_{j})
\end{align*}
Where $\circ $ the product is the product using the Jacobian (see equation (\ref{eq:jacobian_product})), for SVF, or the matrix product.

For $k=2$ it become
\begin{align*}
\text{BCH}^{2}(\mathbf{u}_{j}, \text{app}(-\mathbf{u}_{j}  \oplus  \mathbf{u} ))
&=
\mathbf{u}_{j} +  \text{app}(-\mathbf{u}_{j}  \oplus  \mathbf{u} ) 
+ \frac{1}{2}[\mathbf{u}_{j},  \text{app}(-\mathbf{u}_{j}  \oplus  \mathbf{u} )] + \\
&+ \frac{1}{12}\Big([\mathbf{u}_{j},  [\mathbf{u}_{j},  \text{app}(-\mathbf{u}_{j}  \oplus  \mathbf{u} )]] +\\
&+ [\text{app}(-\mathbf{u}_{j}  \oplus  \mathbf{u} ),  [\text{app}(-\mathbf{u}_{j}  \oplus  \mathbf{u} ) ,\mathbf{u}_{j}  ]] \Big)\\
&=
\mathbf{u}_{j} + \exp(-\mathbf{u}_{j})\circ \varphi  - e 
+ \frac{1}{2}[\mathbf{u}_{j},  \exp(-\mathbf{u}_{j})\circ \varphi  - e ] + \\
&+ \frac{1}{12}\Big([\mathbf{u}_{j},  [\mathbf{u}_{j},  \exp(-\mathbf{u}_{j}) \circ\varphi  - e ]]+\\
&+ [\exp(-\mathbf{u}_{j})\circ \varphi  - e ,  [\exp(-\mathbf{u}_{j}) \circ\varphi - e  ,\mathbf{u}_{j}  ]] \Big)\\
\end{align*}

When considering $k = \infty$ and so the analytic solution of the Lie log-composition provided by the BCH formula, the following theorem, presented in \cite{Bossa:08}, provides an error bound:
\begin{theorem}[Bossa]\label{th:bossa}
	The iterative algorithm 
	\begin{equation}
	\begin{cases}
	\mathbf{u}_0 = 0 \\
	\mathbf{u}_{j+1} %= \text{BCH}(\mathbf{u}_{j}, \text{app}(-\mathbf{u}_{j}  \oplus  \mathbf{u} ))
	                          = \mathbf{u}_{j} \oplus  \text{app}(-\mathbf{u}_{j}  \oplus  \mathbf{u} )
	\end{cases}
	\end{equation}
	converges to $\mathbf{v}$ with error $\delta_n \in \mathbb{G}$, where
	\begin{align*}
	\delta_{n} := \log(\exp(\mathbf{v})\circ \exp(-\mathbf{v}_{n})) \in O(\euclideanMetric{p - e}^{2^{n}})
	\end{align*}
\end{theorem}

We observe that this upper limit can be computed only when a closed-form for the Lie log-composition is available, as for example $\mathfrak{se}(2)$.

% % % % % % % % % % % % % % % % % % % % % % % % % % % % % % % % % % % % % %
% % 
% % % % % % % % % % % % % % % % % % % % % % % % % % % % % % % % % % % % % % 
\subsection{Parallel Transport Strategy}

If we apply the parallel transport method for the computation of the Lie log-composition, we obtain another version of the algorithm:
\begin{equation}\label{eq:bossa_parallel_strategy}
\begin{cases}
\mathbf{u}_0 = \mathbf{0} \\
\mathbf{u}_{j+1} 
= 
\mathbf{u}_{j} 
+ 
\exp(\frac{\mathbf{u}_{j}}{2}) \circ \exp\Big(   \text{app}(-\mathbf{u}_{j}  \oplus  \mathbf{u} ) \Big)
\circ 
\exp(-\frac{\mathbf{u}_{j}}{2}) - e
\end{cases}
\end{equation}
That is computed as:
\begin{align*}
\mathbf{u}_{j+1} 
= 
\mathbf{u}_{j} 
+ 
\exp(\frac{\mathbf{u}_{j}}{2}) \circ \exp\Big(   \exp(-\mathbf{u}_{j}) \circ\varphi  - e \Big)
\circ 
\exp(-\frac{\mathbf{u}_{j}}{2}) - e
\end{align*}

We notice that mixing the operation of composition, sum and scalar product makes sense when the involved vectors are \emph{small enough}, as stated in \ref{se:space_of_approximation}. 
Analytical computation of an upper bound error is not straightforward in this case. 

% % % % % % % % % % % % % % % % % % % % % % % % % % % % % % % % % % % % % %
% 
% % % % % % % % % % % % % % % % % % % % % % % % % % % % % % % % % % % % % % 
\subsection{Symmetrization Strategy}
The algorithm \ref{eq:bossa_reformulated} could have been reformulated alternatively as $\mathbf{u}_{j+1} =    \text{app}(\mathbf{u} \oplus  - \mathbf{u}_{j}  ) \oplus \mathbf{u}_{j}$. The Lie log-composition is not symmetric therefore the two version in some cases may not return the same value. We consider
\begin{equation}\label{eq:bossa_symmetric}
\begin{cases}
\mathbf{u}_0 = 0 \\
\mathbf{u}_{j+1} = \mathbf{u}_{j} \oplus 
\frac{1}{2}
\big(  
\text{app}(-\mathbf{u}_{j}  \oplus  \mathbf{u} )
+
\text{app}(\mathbf{u} \oplus  - \mathbf{u}_{j}  )
\big)
\end{cases}
\end{equation}
that using the BCH approximation of degree $1$, becomes:
\begin{equation}
\begin{cases}
\mathbf{u}_0 = 0 \\
\mathbf{u}_{j+1} 
=
\mathbf{u}_{j} +  
\frac{1}{2}
\big(  
\exp(-\mathbf{u}_{j}) \circ \varphi ) - e
+
\varphi\circ\exp(-\mathbf{u}_{j}) - e
\big)
\end{cases}
\end{equation}
Experimental results of the methods presented in this section are show in section \ref{se:results_lie_logarithm_computations}.
