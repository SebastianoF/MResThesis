\chapter{Experimental Results}\label{ch:results}

\begin{flushright}
	\emph{There, I’ve done my best! If this won’t suit I shall have to wait till I can do better. \\ -Jo}
\end{flushright}

zz write chapter intro, how we did it!

xxx Statistics on anatomies: \\
xxx It is of fundamental importance to have the possibility to going from an element of a group of spatial transformations  to a tangent space, in which each vector corresponds to the tangent vector field that this transformation causes on the space. It makes possible to lean on the group structure a structure of vector space, which implies the possibility to compute statistics on the group of transformation as well as compose velocity fields in the tangent space passing through the corresponding transformation (both of them are made possible thanks to the local bijection between the Lie group and the Lie algebra ). \\

% % % % % % % % % % % % % % % % % % % % % % % % % % % % % % % % % % % % % %
% % SUBSECTION
% % % % % % % % % % % % % % % % % % % % % % % % % % % % % % % % % % % % % % 
\section{Group composition at the Service of Image Registration}

xxx Sum up: log-composition in diffeomorphic image registration.\\


% % % % % % % % % % % % % % % % % % % % % % % % % % % % % % % % % % % % % %
% % % % % % % % % % % % % % % % % % % % % % % % % % % % % % % % % % % % % %
% % SECTION
% % % % % % % % % % % % % % % % % % % % % % % % % % % % % % % % % % % % % %
% % % % % % % % % % % % % % % % % % % % % % % % % % % % % % % % % % % % % %


\section{Experimental Results}

\subsection{Log-Compositions Performance for Toy examples and Patient Images}


\subsection{Logarithm Computation using Log-composition}





\section{Further Research and Conclusion}\label{ch:conclusions}

%\begin{flushright}
%	\emph{There, I’ve done my best! If this won’t suit I shall have to wait till I can do better. \\ -Jo}
%\end{flushright}