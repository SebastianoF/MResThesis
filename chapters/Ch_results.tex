\chapter{Experimental Results}\label{ch:results}

\begin{flushright}
	\emph{\lq\lq A victory is twice itself when the achiever brings home full numbers.\rq\rq \\
		       \emph{Much ado about nothing}, Leonato, scene 1.}
\end{flushright}

xxx Sum up what has been done

xxx Explored tools for statistics of diffeomorphisms.



xxx Statistics on anatomies: \\
xxx It is of fundamental importance to have the possibility to going from an element of a group of spatial transformations  to a tangent space, in which each vector corresponds to the tangent vector field that this transformation causes on the space. It makes possible to lean on the group structure a structure of vector space, which implies the possibility to compute statistics on the group of transformation as well as compose velocity fields in the tangent space passing through the corresponding transformation (both of them are made possible thanks to the local bijection between the Lie group and the Lie algebra ). \\

% % % % % % % % % % % % % % % % % % % % % % % % % % % % % % % % % % % % % %
% % SUBSECTION
% % % % % % % % % % % % % % % % % % % % % % % % % % % % % % % % % % % % % % 
\section{Group composition at the Service of Image Registration}

xxx Sum up: log-composition in diffeomorphic image registration.\\


% % % % % % % % % % % % % % % % % % % % % % % % % % % % % % % % % % % % % %
% % % % % % % % % % % % % % % % % % % % % % % % % % % % % % % % % % % % % %
% % SECTION
% % % % % % % % % % % % % % % % % % % % % % % % % % % % % % % % % % % % % %
% % % % % % % % % % % % % % % % % % % % % % % % % % % % % % % % % % % % % %


\section{Experimental Results}

\subsection{Log-Compositions Performance for Toy examples and Patient Images}


\subsection{Logarithm Computation using Log-composition}





\section{Further Research and Conclusion}\label{ch:conclusions}


Seeing only the results, this one-year research can be considered much ado about nothing. 
