\chapter{Experimental Results}\label{ch:results}

\begin{flushright}
	\emph{\lq\lq A victory is twice itself when the achiever brings home full numbers.\rq\rq \\
		       \emph{Much ado about nothing}, Leonato, scene 1.}
\end{flushright}

\vspace{0.6cm}

In {\bf chapter \ref{ch:introduction}} the concept of log-composition is introduced, emphasizing its implications in medical imaging as a tool utilized in diffeomorphic registration and in the computation of the logarithm in the log-Euclidean framework. 
{\bf Chapter \ref{ch:tools}} is devoted to the introduction of the underpinning mathematical theory: it defines formally the log-composition and presents three numerical methods for its computation:
\begin{enumerate}
	\item Truncated BCH formula of degree $k=1, \frac{3}{2}, 2, 3$ - \emph{equation \ref{eq:bch_definition}}.
	\item Taylor expansion - \emph{equation \ref{eq:taylor}}.
	\item Parallel transport - \emph{equation \ref{eq:parallel_transport}}.
\end{enumerate}
To evaluate their performance, {\bf chapter \ref{ch:spatial_transformations}} reminds two groups of transformations:
\begin{enumerate}
	\item The finite dimensional Lie group of euclidean transformation SE(2) - \emph{section \ref{se:rigid_body_transformations}}
	\item The infinite dimensional Lie group diffeomorphisms, set of images of SVF through the Lie exponential map - \emph{section \ref{se:svf}}
\end{enumerate}
For each of these specific groups it presents as well the numerical methods for the computation log-composition presented in the previous chapter for the general case.
{\bf Chapter \ref{ch:log_algorithm}} presents an algorithm for the computation of the Lie logarithm \cite{Bossa:08}. 
Thanks to the fact that this important piece in the jigsaw puzzle of the log-euclidean framework can be reformulated in term of the log-composition, it is possible to use numerical methods introduced:
\begin{enumerate}
	\item Truncated BCH formula of degree $k=1, 1.5, 2, 3$ - \emph{equation \ref{eq:bossa_bch_strat}}.
	\item Parallel transport - \emph{equation \ref{eq:bossa_parallel_strategy}}.
	\item Symmetric parallel transport - \emph{equation \ref{eq:bossa_symmetric}}.
\end{enumerate}

This last chapter presents some of the results of the numerical methods investigated.
Computations are performed with a software written in Python (available on gitlab at ... add ref), based on numerous libraries as well as on the framework NiftyBit, implemented by Pancaj Daga.

% % % % % % % % % % % % % % % % % % % % % % % % % % % % % % % % % % % % % %
% % % % % % % % % % % % % % % % % % % % % % % % % % % % % % % % % % % % % % 
% % % % % % % % % % % % % % % % % % % % % % % % % % % % % % % % % % % % % % 
\section{Log-composition for $\mathfrak{se}(2)$}

There are several norms in the space of $3\times 3$ squared matrices that can be inherited by the group $SE(2)$ and the Lie algebra $\mathfrak{se}(2)$ when represented by matrices. For our tests we considered the tangent space $\mathfrak{se}(2)$ with the inherited Frobenius norm:
\begin{align*}
\euclideanMetric{(\theta,dt^{x},dt^{y})}_{\text{fro}} = \sqrt{2\theta^{2} + (dt^{x})^2 + (dt^{y})^2} 
\qquad
(\theta,dt_{x},dt_{y}) \in \mathfrak{se}(2)
\end{align*}
Numerical test shown that for the studied cases, no qualitative differences are detected if choosing instead the $L^{2}$ norm.
 %
 \begin{figure}[!ht]
 	\hspace{-2cm}
 	\includegraphics[scale=0.54]{figures/se2_image_scale.png}
 	\caption{Log composition in $\mathfrak{se}(2)$. Comparisons between methods. TODO}
 	\label{fig:se2_image_scale}
 \end{figure}
 %
 
% % % % % % % % % % % % % % % % % % % % % % % % % % % % % % % % % % % % % %
% % % % % % % % % % % % % % % % % % % % % % % % % % % % % % % % % % % % % % 
\subsection{Methods and Results}

Two sets of $3000$ transformations of elements in $\mathfrak{se}(2)$ are randomly sampled with increasing norms in the interval $[0.1, 2.0]$. This interval is divided into 6 segments delimited by $ I = \text{linspace}([0.1, 2.0], 7)$ and for each couple of subintervals $[I(n_0), I(n_0+1)]$, $[I(n_1), I(n_1+1)]$ two sets of $500$ transformations $\{ dr_{0}^{(j)}\}_{j=1}^{500}$, $\{ dr_{1}^{(j)} \}_{j=1}^{500}$ having norms belonging to the respective intervals are sampled:
\begin{align*}
j=1,...,500 \qquad &\qquad  n_0, n_1 = 1,,...,6 \\
\euclideanMetric{dr_{0}^{(j)}}_{\text{fro}} &\in [I(n_0), I(n_0+1)] \\
\euclideanMetric{dr_{1}^{(j)}}_{\text{fro}} &\in [I(n_1), I(n_1+1)]  
\end{align*} 
If $N$ is one of the numerical methods presented in section \ref{se:rigid_body_transformations} for the computation of the log-composition - $\text{BCH}^{0}, \text{BCH}^{1}, \text{BCH}^{2}, \text{Tl}, \text{pt}$ - 
then the error between the ground truth and the approximation provided by one of these numerical methods is given by
\begin{align*}
\text{Error}(dr_{0},dr_{1},N) 
:= 
\euclideanMetric{dr_{0}^{(j_0)} \oplus dr_{1}^{(j_1)} 
- 
N(dr_{0}, dr_{1}) }_{\text{fro}} 
\end{align*}

In figure \ref{fig:se2_image_scale}, each of the figure corresponds to a different method and each of the grade scale is the value computed with the function:
\begin{align*}
f(n_0,n_1,N) 
=
 \mathbb{E}\Big(
  \{ 
  \text{Error}(dr_{0}^{j},dr_{1}^{j},N) 
  \}_{j=1}^{500}
  \Big)
\end{align*}
Where the norm of $dr_{0}^{(j)}$ belongs to the interval $[I(n_0), I(n_0+1)]$ and the norm of 
$dr_{1}^{(j)}$ belongs to $[I(n_1), I(n_1+1)]$, and where $\mathbb{E}$ is the mean value.\\
The data indicated by the gray arrows in each plot corresponds are showed in the box-plot \ref{fig:se2_boxplot}
%
\begin{figure}[!ht]
	\hspace{-1.5cm}
	\includegraphics[scale=0.54]{figures/se2_boxplot.png}
	\caption{Log composition in $\mathfrak{se}(2)$. comparisons between all the methods. TODO}
	\label{fig:se2_boxplot}
\end{figure}
%

From these results we can see that the second truncation error of the BCH formula provides the best result even when the norm of the both transformation are around $2.0$ (the unit of measure is the same as the measure chosen for the translation or the rotation: it can be inches, cm, pixel, ...). Except the numerical method involving the parallel transport, all of the errors are symmetric in function of the norm of the input vectors. \\
Parallel transport method provides better results when $dr_1$ has little norm.


% % % % % % % % % % % % % % % % % % % % % % % % % % % % % % % % % % % % % %
% % % % % % % % % % % % % % % % % % % % % % % % % % % % % % % % % % % % % %
% % % % % % % % % % % % % % % % % % % % % % % % % % % % % % % % % % % % % % 
% % % % % % % % % % % % % % % % % % % % % % % % % % % % % % % % % % % % % % 
\section{Log-composition for SVF}


\begin{figure}[!ht]
	\hspace{-0.8cm}
	\includegraphics[scale=0.72]{figures/svf_gaussian_smoothing_effects.png}
	\caption{Random generated vector field before and after the Gaussian smoother: in the first row a random generated vector field of dimension $50\times 50 \times 2$ where the value at each pixel are sampled from a random variable with normal distribution of mean $0$ and sigma $4$. The second row shows the same random vector field after a Gaussian smoothing of sigma $2$ (the code is based on the scipy library ndimage.filters.gaussian\textunderscore filter). In the last column shows the quiver of the vector field in the squared subregion of size $10\times 10$ at the point $(20,20)$. From the colorscale it is also possible to see that the values distribution of the filtered image is not anymore symmetric. }
	\label{fig:svf_gaussian_smoothing_effects}
\end{figure}


\begin{figure}[!ht]
	\hspace{-1.5cm}
	\includegraphics[scale=0.54]{figures/SVF_sigma_means_comparisons.png}
	\caption{Norm of random generated of SVF with initial standard deviation $\sigma_{\text{init}}$ (on the x-axis) and Gaussian filter with standard deviation $\sigma_{\text{gf}}$ (different colors). On the left is shown the the Frobenius norm computed on the SVF in the Lie algebra, while on the right the same norm is computed after the exponentiations. In this second case, the norm refers to the norm of the matrix data structure (DS-norm) utilized to parametrize the SVF. Each dot represents the mean of the norm of $10$ an SVF randomly generated with the parameters indicated on the axes and in the legend. We observe that the exponential bend the shape of the random SVF when the Gaussian filter is $0$ (thus we talk about an improper SVF). The decrease in the slope when $\sigma_{\text{gf}}=0$ do not appears for any other value.}
	\label{fig:SVF_sigma_means_comparisons}
\end{figure}

\begin{figure}[!ht]
	\hspace{-1cm}
	\includegraphics[scale=0.6]{figures/SVF_image_scale_bracket_versus_gaussian.png}
	\caption{Norm of the Lie bracket as direct consequence of the sigma of the Gaussian smoother of respective SVF. Here put every other data!}
	\label{fig:SVF_image_scale_bracket_versus_gaussian}
\end{figure}


% % % % % % % % % % % % % % % % % % % % % % % % % % % % % % % % % % % % % %
% % % % % % % % % % % % % % % % % % % % % % % % % % % % % % % % % % % % % % 
\subsection{Methods: random generated SVF, and norm comparisons.}
We created synthetic random matrices in $\mathfrak{se}(2)$. We considered the difference between the log composition computed with the closed form \ref{eq:log_composition_se2_closed_form}. 
A sample of $500$ couples $(dr_0,dr_1)$ of elements in $\mathfrak{se}(2)$ are created, 

norm in the lie algebra

norm in the lie group


Norm of a group. Inherited, start form the previous case!

How the error is computed in the group and why


\begin{figure}[!ht]
	\hspace{-1.5cm}
	\includegraphics[scale=0.6]{figures/SVF_bch_parallel_transport.png}
	\caption{Log-composition for SVF computed using numerical methods of truncated BCH of degree 0,1 and parallel transport.}
	\label{fig:SVF_bch_parallel_transport}
\end{figure}


\begin{figure}[!ht]
	\hspace{-2.1cm}
	\includegraphics[scale=0.53]{figures/SVF_image_scale.png}
	\caption{Log-composition for SVF; the operation $\mathbf{u}_0\oplus \mathbf{u}_1$ is computed using numerical methods of truncated BCH of degree 0,1 and parallel transport. Respective standard deviation of the random generated SVF given by $\sigma_0$ and $\sigma_1$, ranges between $0.3$ and $2.0$ for $\sigma_0$ and between $0.2$ and $2.0$ for $\sigma_1$. Each value in the image scale is the mean of $10$ results of the log-computation of random SVF generated with given standard deviation. For lower values of $\mathbf{u}_1$, that in the image registration algorithms are given by the update, parallel transport method and truncated BCH of degree 1 have comparable results. We can also notice that for truncated BCH methods the results are symmetric, while for parallel transport, as expected from the formula, results are not symmetric respect to the size of the input vectors. }
	\label{fig:SVF_image_scale}
\end{figure}


\begin{figure}[!ht]
	\hspace{-2cm}
	\includegraphics[scale=0.6]{figures/SVF_boxplot.png}
	\caption{Log-composition for SVF computed using numerical methods of truncated BCH of degree 0,1 and parallel transport, represented in a boxplot.}
	\label{fig:SVF_boxplot}
\end{figure}










% % % % % % % % % % % % % % % % % % % % % % % % % % % % % % % % % % % % % %
% % % % % % % % % % % % % % % % % % % % % % % % % % % % % % % % % % % % % % 
\subsection{Results}


% % % % % % % % % % % % % % % % % % % % % % % % % % % % % % % % % % % % % %
% % % % % % % % % % % % % % % % % % % % % % % % % % % % % % % % % % % % % % 
% % % % % % % % % % % % % % % % % % % % % % % % % % % % % % % % % % % % % % 
\section{Log-Algorithm for SVF}

% % % % % % % % % % % % % % % % % % % % % % % % % % % % % % % % % % % % % %
% % % % % % % % % % % % % % % % % % % % % % % % % % % % % % % % % % % % % % 
\subsection{Methods}

% % % % % % % % % % % % % % % % % % % % % % % % % % % % % % % % % % % % % %
% % % % % % % % % % % % % % % % % % % % % % % % % % % % % % % % % % % % % % 
\subsection{Results}


% % % % % % % % % % % % % % % % % % % % % % % % % % % % % % % % % % % % % %
% % % % % % % % % % % % % % % % % % % % % % % % % % % % % % % % % % % % % % 
\section{Empirical Evaluations of Computational Time}


% % % % % % % % % % % % % % % % % % % % % % % % % % % % % % % % % % % % % %
% % % % % % % % % % % % % % % % % % % % % % % % % % % % % % % % % % % % % %
% % % % % % % % % % % % % % % % % % % % % % % % % % % % % % % % % % % % % % 
\section{Conclusions and Further Research}\label{ch:conclusions}


Considering only the results, this one-year research can be considered much ado about nothing, but...\\
Computational time...!

Starting from the definition of Lie log-group of diffeomorpshisms $(\mathfrak{g} , \oplus)$, to have an algebraic definition of this approximation, we can consider its quotient over the ideal generated by $(\text{ad}_{\mathbf{u}}^{m}, \text{ad}_{\mathbf{u}}^{n})$, which provides the group $(\quotient{\mathfrak{g}}{(\text{ad}_{\mathbf{u}}^{m}, \text{ad}_{\mathbf{u}}^{n})}, \oplus)$. Further investigations in this direction is not prosecuted.


The BCH is proved only when the exp and log can be expressed in power series, so when the Lie group and the Lie algebra involved belongs to the same bigger group. This is not the case of the infinite dimensional Lie group of diffeomorphisms,