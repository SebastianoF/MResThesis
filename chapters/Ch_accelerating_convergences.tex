% % % % % % % % % % % % % % % % % % % % % % % % % % % % % % % % % % % % % %
% % % % % % % % % % % % % % % % % % % % % % % % % % % % % % % % % % % % % %
% % SECTION
% % % % % % % % % % % % % % % % % % % % % % % % % % % % % % % % % % % % % %
% % % % % % % % % % % % % % % % % % % % % % % % % % % % % % % % % % % % % %

\chapter{Accelerating Convergences Series}\label{ch:accelerating}

xxx Think about it after 14 of may!

Space of the series of elements of $\mathfrak{g}$
\begin{align*}
S(\mathfrak{g}) = \{ \sum_{j=0}^{\infty} \mathbf{u}_{j} \mid \mathbf{u}_{j} \in \mathfrak{g}  \}
\end{align*}
Series generate by 
\begin{align*}
S(\mathbf{u}) =\sum_{j=0}^{\infty} \mathbf{u}^{j} 
\end{align*}
\begin{align*}
\exp(\mathbf{u}) = S_1  \cdot S(\mathbf{u}) 
\end{align*}
where $S_1 = \sum_{k=0}^{\infty} (\frac{1}{k!})$ in the space of coefficient series $\cdot $ is the (infinite) scalar product in the space of series.\\
$k$-th series truncation:
\begin{align*}
S^{k}(\mathfrak{g})  = \{ \sum_{j=0}^{k} \mathbf{u}_{j} \mid \mathbf{u}_{j} \in \mathfrak{g}  \}
\end{align*}
This notation may make sense as a starting point to define $\exp(\mathbf{u})$. 
The restriction to the first order truncation of the exp is the starting point the numerical approximation
\begin{align*}
\exp(\mathbf{u}) = 1 + \mathbf{u} \in S^{1}(\mathfrak{g}) 
\end{align*}