
% % % % % % % % % % % % % % % % % % % % % % % % % % % % % % % % % % % % % %
% % % % % % % % % % % % % % % % % % % % % % % % % % % % % % % % % % % % % %
% % % % % % % % % % % % % % % % % % % % % % % % % % % % % % % % % % % % % % 
\chapter{Conclusions}\label{ch:conclusions}

In this work we have formally defined the mathematical concept of the Lie log-composition and presented the limitations of the numerical methods for its computation obtained with truncations of the BCH formula. These limitations are the starting point of the research for BCH-free numerical methods. 
One method, based on the geometrical concept of parallel transport, is presented here for the first time, and formally proved for generalized Lie algebra.

This method is compared with the truncated BCH, both for elements belonging to the Lie algebra of rigid body transformations (where also another BCH-free numerical method based on the Taylor expansion is available) and for stationary velocity fields (SVF).

The possible applications of efficient numerical methods for the computation of the Lie log-composition in medical imaging are listed in section \ref{se:applications_log_com_in_med}. The computation of the Lie logarithm based on the algorithm presented in \cite{bossa2008new}, is investigated in chapter \ref{ch:log_algorithm}.

Results show that the Lie log-composition computed with the parallel transport method improves the $\text{BCH}^0$, and gets, in general close to the $\text{BCH}^1$. At a higher computational cost, the parallel transport method avoids the computation of the Jacobian involved in the $\text{BCH}^1$.

We can conclude that, for medical applications, it is not recommendable to use numerical approximations based on $\text{BCH}^k$ for $k\geq 2$. The method that provides the best results both on synthetic and real data on a wider range of SVF, is the $\text{BCH}^1$. The parallel transport is the first choice (and the unique at the moment) when it is preferable to avoid the computation of the Lie bracket. 


% % % % % % % % % % % % % % % % % % % % % % % % % % % % % % % % % % % % % %
% % % % % % % % % % % % % % % % % % % % % % % % % % % % % % % % % % % % % % 
\section{Further Researches}\label{se:further_research}

Since this thesis has both numerical and theoretical aspects, improvements could be reached getting a better grasp of both these sides.

\subsection{Numerical Computations} 

To compute the Lie exponential and the Lie logarithm, we always used the scaling and squaring and the inverse scaling and squaring algorithms, as originally proposed by Arsigny in $2006$ \cite{arsigny2006log}. There are other options available that, when applied to the Lie log-composition, could improve efficiency. These are based for example on the Euler method, the midpoint method, the modified Euler method and Runge-Kutta of order $4$. Investigations in this direction are currently in progress.

The proof of the convergence for the BCH in Dynkin's paper is based on the power series expansion of the exponential and of the logarithm (equation (\ref{eq:exp_as_inf_sum}) and (\ref{eq:log_as_inf_sum}))
but these expansions, unless using the function $\mathcal{V}$ proposed in section \ref{subse:bigger_algebra} are not well defined, and have never been proved for the Lie algebra of diffeomorphisms.
Numerical tests, on which we are currently working, seem to show that the Lie exponential computed with the expansion in power series converges for SVF. In this case we utilized the product $\cdot$ that appears in equation \ref{eq:jacobian_product}, $\mathbf{u}\cdot \mathbf{v} = J_{\mathbf{u}} \mathbf{v}$.

\subsection{Theoretical Formulae}

Certainly not all of the possibilities provided by the application of the concept of parallel transport for the computation of the Lie log-composition have been exploited. The formula (\ref{eq:parallel_transport}), presented at the end of chapter \ref{ch:tools}, can still be improved, for example finding strategies to reduce the number of underpinning assumptions and introducing some numerical techniques to compute the Affine exponential. When this tool is available, formula (\ref{eq:parallel_transport}) could be reformulated as
\begin{align*}
\mathbf{u}\oplus \mathbf{v}
&\simeq
\exp_{\exp(\mathbf{u} )}(\mathbf{v} ) - e
\end{align*} 
or, in a symmetric version,  as
\begin{align*}
\mathbf{u}\oplus \mathbf{v}
&\simeq
\frac{1}{2}
\big(
\exp_{\exp(\mathbf{u} )}(\mathbf{v} ) 
+
\exp_{\exp(\mathbf{v} )}(\mathbf{u} )
\big)
-  e
\end{align*} 

Another BCH-free formula, not based on the parallel transport, could be obtained by finding a way to extend the Taylor expansion proposed for $SE(2)$ in section \ref{se:rigid_body_transformations} to SVF.

A third one, on which  preliminary tests showed promising results, is obtained using the accelerating convergence series \cite{cohen2000convergence} on the series expansion of the Lie exponential and the Lie logarithm. This is based on the assumption that Lie logarithm and Lie exponential are analytic also for SVF.

