\chapter{Numerical Approximation of the Log-Composition}\label{ch:application_log_composition}

\begin{flushright}
	\emph{We believe that we know something about the things themselves when we speak of trees, colors, snow, and flowers; and yet we possess nothing but metaphors for things — metaphors which correspond in no way to the original entities.} \\ -Nietzsche, \emph{On Truth and Lies in extra-moral sense.}
\end{flushright}


xxx Statistics on anatomies: \\
xxx It is of fundamental importance to have the possibility to going from an element of a group of spatial transformations  to a tangent space, in which each vector corresponds to the tangent vector field that this transformation causes on the space. It makes possible to lean on the group structure a structure of vector space, which implies the possibility to compute statistics on the group of transformation as well as compose velocity fields in the tangent space passing through the corresponding transformation (both of them are made possible thanks to the local bijection between the Lie group and the Lie algebra ). \\

% % % % % % % % % % % % % % % % % % % % % % % % % % % % % % % % % % % % % %
% % SUBSECTION
% % % % % % % % % % % % % % % % % % % % % % % % % % % % % % % % % % % % % % 
\section{Group composition at the Service of Image Registration}

xxx Log-composition in diffeomorphic image registration.\\


where $p_1 \in \mathbb{G}$, $p_2 = \exp_{p_1}(\mathbf{v}_{1})$ and Affine exponential and Affine logarithm are considered. Posing $p_3 = \exp_{p_2}(\mathbf{v}_{2})$ the second offset group composition is the tangent vector in $p_1$ that corresponds to the composition between $p_1$ and $p_2$, i.e. $\log_{p_1}(p_3 \circ p_2)$. \\

% % % % % % % % % % % % % % % % % % % % % % % % % % % % % % % % % % % % % %
% % SUBSECTION
% % % % % % % % % % % % % % % % % % % % % % % % % % % % % % % % % % % % % % 
\section{BCH formula to compute the Log-composition}

The BCH formula is the exact solution to the Log-composition. 
It consists of an infinite series of Lie bracket whose asymptotic behaviour cannot be predicted only from the coefficient of each nested Lie bracket term. It can be practically computed using its \emph{approximation of degree} $k$ defined as the sum of the BCH terms having no more than $k$ nested Lie bracket. For example:
\begin{align*}
BCH^{0}(\mathbf{u},\mathbf{v}) &= \mathbf{u} + \mathbf{v} \\
BCH^{1}(\mathbf{u},\mathbf{v}) &=  \mathbf{u} + \mathbf{v} + \frac{1}{2}[\mathbf{u},\mathbf{v}] \\
BCH^{2}(\mathbf{u},\mathbf{v}) &=  \mathbf{u} + \mathbf{v} + \frac{1}{2}[\mathbf{u},\mathbf{v}] + \frac{1}{12}([\mathbf{u},[\mathbf{u},\mathbf{v}]] + [\mathbf{v},[\mathbf{v},\mathbf{u}]])
\end{align*}
These numerical approximations of the group composition leave the difficulty of managing the problem of the error carried by each term. In some cases the increase of the degree of the BCH approximation do not necessarily implies a decrease in error: \\
.... Add an example in which this happens.

We present other ways to compute the Lie group composition in the following subsections.

% % % % % % % % % % % % % % % % % % % % % % % % % % % % % % % % % % % % % %
% % SUBSECTION
% % % % % % % % % % % % % % % % % % % % % % % % % % % % % % % % % % % % % % 
\section{Accelerating convergences applied to the Log-composition}

....

% % % % % % % % % % % % % % % % % % % % % % % % % % % % % % % % % % % % % %
% % SUBSECTION
% % % % % % % % % % % % % % % % % % % % % % % % % % % % % % % % % % % % % % 
\section{Taylor expansion to compute the Log-composition}

....


% % % % % % % % % % % % % % % % % % % % % % % % % % % % % % % % % % % % % %
% % SUBSECTION
% % % % % % % % % % % % % % % % % % % % % % % % % % % % % % % % % % % % % % 
\section{Parallel transport to compute the Log-composition}

....Here will be made the strong assumption according to which the parallel transport defined for the finite dimensional case, works also in the infinite dimensional case....

