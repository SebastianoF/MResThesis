\chapter{Numerical Approximation to Compute the Lie logarithm}\label{ch:lie_log_computation}


% % % % % % % % % % % % % % % % % % % % % % % % % % % % % % % % % % % % % %
% % SUBSECTION
% % % % % % % % % % % % % % % % % % % % % % % % % % % % % % % % % % % % % % 
\section{Exponential and Logarithm Approximation with Truncated Power Series}



% % % % % % % % % % % % % % % % % % % % % % % % % % % % % % % % % % % % % %
% % SUBSECTION
% % % % % % % % % % % % % % % % % % % % % % % % % % % % % % % % % % % % % % 
\section{A reformulation of the Bossa Algorithm using Log-composition}



Having explored some methods to evaluate the BCH formula we can use them to find an approximation of the logarithm function of an element of $\mathbb{G}$. 
Given $p \in \mathbb{G}$ the goal is to find $\mathbf{u}$ such that $\exp(\mathbf{u})$ is the best possible approximation of $p$.  

The first way to find a solution is via the algorithm presented in \cite{Bossa:08}, it reduces the problem of the computation of logarithm to the problem of the computation of the group composition. \\
If $p= \exp(\mathbf{v})$ for any $\mathbf{v} \in \mathfrak{g}$, near the identity we can write:
\begin{align*}
p= \exp(\mathbf{v}) &= (\exp(\mathbf{v})\circ \exp(-\mathbf{v}))\circ \exp(\mathbf{v})\\
&= \exp(\mathbf{v})\circ (\exp(-\mathbf{v})\circ p)\\
&= \exp(\mathbf{v})\circ \exp(\delta \mathbf{v})\\
&\approx \exp(\mathbf{v})\circ \exp(\tilde{\delta} \mathbf{v})
\end{align*}
Where $\tilde{\delta} \mathbf{v}$, as we are going to see, turns out to be $ \exp(-\mathbf{v}) \circ p - e$, for $e$ identity transformation of the Lie group.
The iterative algorithm is then
\begin{equation}\label{eq:bossa_strat}
\begin{cases}
\mathbf{v}_0 = 0 \\
\mathbf{v}_{n} = \text{BCH}^{k}(\mathbf{v}_{n-1},\tilde{\delta} \mathbf{v}_{n-1})
\end{cases}
\end{equation}
for some degree $k$ of approximation.
\begin{align*}
\tilde{\delta} \mathbf{v}_{n-1} =  \exp(-\mathbf{v}_{n-1} )\circ \Phi - e
\end{align*}
%
\begin{proof}  
	Let $\mathbf{v}_{0}$ be an element of $\mathfrak{g}$ in some sense close to $\mathbf{v}$ then:
	\begin{align*}
	p = \exp(\mathbf{v}) &= \exp(\mathbf{v}_{0})\circ (\exp(-\mathbf{v}_{0})\circ p)
	\end{align*}
	We define $\delta \mathbf{v}_{0} \in\mathfrak{g}$ as $\delta \mathbf{v}_{0} = \exp(-\mathbf{v}_{0})\circ p$. Then
	\begin{align*}
	p &= \exp(\mathbf{v}_{0})\circ \exp(\delta \mathbf{v}_{0}) \\
	\exp(V) &= \exp(\mathbf{v}_{0})\circ \exp(\delta \mathbf{v}_{0}) \\
	\mathbf{v} &= \log(\exp(\mathbf{v}_{0})\circ \exp(\delta \mathbf{v}_{0}))\\
	\mathbf{v} &\simeq \text{BCH}^{k}(\mathbf{v}_{0},\delta \mathbf{v}_{0})
	\end{align*}
	We approaching the tangent vector $\mathbf{v}$ using an iterative algorithm based on the $\text{BCH}$ formula and the lemma \ref{le:taylorlemma}.
	\begin{align*}
	\exp(\delta \mathbf{v}_{0}) \approx e + \delta \mathbf{v}_{0} \Longrightarrow \delta \mathbf{v}_{0} \approx  \exp(\delta \mathbf{v}_{0}) - e
	\end{align*}
	Having $\mathbf{v}_{0}$ as our initial value we define
	\begin{align*}
	\tilde{\delta} \mathbf{v}_{0} := \exp(\delta \mathbf{v}_{0}) - e
	\end{align*}
	Using $p = \exp(\mathbf{v}_{0})\circ \exp(\delta \mathbf{v}_{0})$ we can say that $\exp(\delta \mathbf{v}_{0}) =  \exp(-\mathbf{v}_{0})\circ p$ and then
	\begin{align*}
	\tilde{\delta} \mathbf{v}_{0} = \exp(-\mathbf{v}_{0})\circ p - e
	\end{align*}
	just by definition. Since $p$ is known we can start our successive approximation, and if we set $\mathbf{v}_{0} = \mathbf{0}$ we end up with the iterative algorithm (\ref{eq:bossa_strat}).
\end{proof}

\begin{theorem}[Bossa]\label{th:bossa}
	The iterative algorithm (\ref{eq:bossa_strat}) converges to $\mathbf{v}$ with error $\delta_n \in \mathbb{G}$, where
	\begin{align*}
	\delta_{n} := log(\exp(\mathbf{v})\circ \exp(-\mathbf{v}_{n})) \in O(\euclideanMetric{p - e}^{2^{n}})
	\end{align*}
\end{theorem}
%\begin{proof}
%	TODO See notebook or \cite{bossa}.
%\end{proof}

% % % % % % % % % % % % % % % % % % % % % % % % % % % % % % % % % % % % % %
% % SUBSECTION
% % % % % % % % % % % % % % % % % % % % % % % % % % % % % % % % % % % % % % 
\section{Parallel Transport Strategy}

We can use the consideration of the previous section to evaluate the BCH formula in this context, and get an approximation for the evaluation of the Log function.
\begin{align*}
\tilde{\delta} \mathbf{v}_{t-1} = \exp(-\mathbf{v}_{t-1})\circ p - e 
\end{align*}
we get the iterative algorithm
\begin{equation}\label{eq:parallel_strategy}
\begin{cases}
\mathbf{v}_0 = \mathbf{0} \\
\mathbf{v}_{t} = \mathbf{v}_{t-1} - \exp(-\frac{\mathbf{v}_{t-1}}{2}) \circ \exp(\delta \mathbf{v}_{t-1}) \circ \exp(\frac{\mathbf{v}_{t-1}}{2}) + e
\end{cases}
\end{equation}

% % % % % % % % % % % % % % % % % % % % % % % % % % % % % % % % % % % % % %
% % SUBSECTION
% % % % % % % % % % % % % % % % % % % % % % % % % % % % % % % % % % % % % % 
\section{Symmetrization Strategy}

The algorithm for the computation of the group logarithm can be improved considering a symmetric version of the underpinning strategy \ref{eq:bossa_strat}. In this version we use the first order approximation of the BCH formula (see equation (\ref{eq:first_order_approx}) in the following proof), compensating with the fact that the symmetrization should decrease the error involved.
It gives birth to the following algorithm:
\begin{equation}\label{eq:sym_strategy}
\begin{cases}
\mathbf{v}_0 = \mathbf{0} \\
\mathbf{v}_{t+1} = \mathbf{v}_{t} + \frac{1}{2}(\tilde{\delta} \mathbf{v}^{L}_{t} +\tilde{\delta} \mathbf{v}^{R}_{t})
\end{cases}
\end{equation}
Where $\tilde{\delta} \mathbf{v}^{R}_{t} = \exp(\mathbf{v})\circ \exp(- \mathbf{v}_{t}) - e$ and $\tilde{\delta} \mathbf{v}^{L}_{t} = \exp(-\mathbf{v}_{t})\circ \exp(\mathbf{v}) - e$.\\
\begin{proof}
	To show why it works we remind that the starting point was
	\begin{align*}
	p= \exp(\mathbf{v}) &=  \exp(\mathbf{v}_{0})\circ \exp(\delta \mathbf{v}_{0})
	\end{align*}
	where $\exp(\delta \mathbf{v}_{0}) = \exp(-\mathbf{v}_{0})\circ p$.\\
	An equivalent starting point would have been $\exp(\mathbf{v}) = \exp(\delta \mathbf{v})\circ \exp(\mathbf{v}_{0})$ for $\exp(\delta \mathbf{v}) = p\circ \exp(-\mathbf{v}_{0})$. \\
	This idea leads to the definition of
	\begin{align*}
	\exp(\delta \mathbf{v}^{R}_{t}) &:= p\circ \exp(- \mathbf{v}_{t}) = \exp(\mathbf{v})\circ \exp(- \mathbf{v}_{t})\\
	\exp(\delta \mathbf{v}^{L}_{t}) &:=  \exp(- \mathbf{v}_{t}) \circ p = \exp(- \mathbf{v}_{t}) \circ \exp(\mathbf{v}) 
	\end{align*}
	It follows that 
	\begin{align*}
	\exp(\mathbf{v}) &= \exp(\mathbf{v}_{0})\circ \exp(\delta \mathbf{v}^{R}_{0})\\
	\exp(\mathbf{v}) &=  \exp(\delta \mathbf{v}^{L}_{0}) \circ \exp(\mathbf{v}_{0})
	\end{align*}
	Using $\exp(\delta \mathbf{v}^{R}_{t}) \approx e + \delta \mathbf{v}^{R}_{t}$ and $\exp(\delta \mathbf{v}^{L}_{t}) \approx e + \delta \mathbf{v}^{L}_{t}$ we can use the following approximation to define the symmetric algorithm:
	\begin{align*}
	\exp(\delta \mathbf{v}^{R}_{t}) &= \exp(\mathbf{v})\circ \exp(-\mathbf{v}_{t})\\
	e + \tilde{\delta} \mathbf{v}^{R}_{t} &= \exp(\mathbf{v})\circ \exp(- \mathbf{v}_{t})\\
	\tilde{\delta} \mathbf{v}^{R}_{t} &= \exp(\mathbf{v})\circ \exp(- \mathbf{v}_{t}) - e
	\end{align*}
	\begin{align*}
	\exp(\delta \mathbf{v}^{L}_{t}) &= \exp(- \mathbf{v}_{t}) \circ \exp(\mathbf{v})\\
	e + \tilde{\delta} \mathbf{v}^{L}_{t} &= \exp(-\mathbf{v}_{t})\circ \exp( \mathbf{v})\\
	\tilde{\delta} \mathbf{v}^{L}_{t} &= \exp(-\mathbf{v}_{t})\circ \exp(\mathbf{v}) - e
	\end{align*}
	Which gives birth to iterative algorithm, for a given initial value $V_0$:  
	\begin{equation}
	\begin{cases}
	\mathbf{v}_0  \\
	\mathbf{v}_{t+1} =\text{BCH}(\mathbf{v}_{t},\tilde{\delta} \mathbf{v}^{R}_{t})
	\end{cases}
	\begin{cases}
	\mathbf{v}_0  \\
	\mathbf{v}_{t+1} = \text{BCH}(\tilde{\delta} \mathbf{v}^{L}_{t}, \mathbf{v}_{t})
	\end{cases}
	\end{equation}
	If follows that
	\begin{align*}
	\mathbf{v}_{t+1} = \frac{1}{2}(\text{BCH}(\tilde{\delta} \mathbf{v}^{L}_{t}, \mathbf{v}_{t}) + \text{BCH}(\mathbf{v}_{t},\tilde{\delta} \mathbf{v}^{R}_{t}))
	\end{align*}
	Taking the first order approximation of the BCH formula:
	\begin{align}\label{eq:first_order_approx}
	BCH(\tilde{\delta} \mathbf{v}^{L}_{t}, \mathbf{v}_{t}) &\approx \tilde{\delta} \mathbf{v}^{L}_{t} + \mathbf{v}_{t}\\
	BCH(\mathbf{v}_{t},\tilde{\delta} \mathbf{v}^{R}_{t}) &\approx \mathbf{v}_{t} + \tilde{\delta} \mathbf{v}^{R}_{t}
	\end{align}
	we get
	\begin{align*}
	\mathbf{v}_{t+1} = \mathbf{v}_{t} + \frac{1}{2}(\tilde{\delta} \mathbf{v}^{L}_{t} + \tilde{\delta} \mathbf{v}^{R}_{t})
	\end{align*}
\end{proof}
We observe that the symmetric approach do not requires to use the BCH formula at each passage, having considered the approximation at the first order of the BCH.\\
We conclude with a formula that relates $\tilde{\delta} \mathbf{v}^{L}_{t}$ with $\tilde{\delta} \mathbf{v}^{R}_{t}$:
\begin{theorem}
	Be $\tilde{\delta} \mathbf{v}^{R}_{t} = \exp(\mathbf{v})\circ \exp(- \mathbf{v}_{t}) - e$ and $\tilde{\delta} \mathbf{v}^{L}_{t} = \exp(-\mathbf{v}_{t})\circ \exp(\mathbf{v}) - e$ as before, then
	\begin{align*}
	\delta \mathbf{v}^{L}_{t} \approx \exp(-\mathbf{v}_{t}) \circ \delta \mathbf{v}^{R}_{t} \circ \exp(\mathbf{v}_{t})
	\end{align*}
\end{theorem}
\begin{proof}
	Since $\exp(\mathbf{v}_{t})\circ \exp(\delta \mathbf{v}^{R}_{t}) \approx exp(\delta \mathbf{v}^{L}_{t}) \circ \exp(\mathbf{v}_{t})$ it follows
	\begin{align*}
	\exp(\delta \mathbf{v}^{R}_{t}) = \exp(-\mathbf{v}_{t})\circ \delta \mathbf{v}^{L}_{t} \circ \exp(\mathbf{v}_{t})
	\end{align*}
	Using $\exp(\delta \mathbf{v}^{R}_{t}) = e + \delta \mathbf{v}^{R}_{t}$ and $\exp(\delta \mathbf{v}^{L}_{t}) = e + \delta \mathbf{v}^{L}_{t}$ we get
	\begin{align*}
	e + \delta \mathbf{v}^{R}_{t} &= \exp(-\mathbf{v}_{t})\circ (e + \delta \mathbf{v}^{L}_{t}) \circ \exp(\mathbf{v}_{t})\\
	\delta \mathbf{v}^{R}_{t} &= \exp(-\mathbf{v}_{t})\circ \delta \mathbf{v}^{L}_{t} \circ \exp(\mathbf{v}_{t})
	\end{align*}
\end{proof}

% % % % % % % % % % % % % % % % % % % % % % % % % % % % % % % % % % % % % %
% % SUBSECTION
% % % % % % % % % % % % % % % % % % % % % % % % % % % % % % % % % % % % % %
\section{Symmetric-Parallel Transport Strategy}
If we are not satisfied to having take only the firs order approximation of the BCH in the equation (\ref{eq:first_order_approx}) we use at this stage the parallel transport in the method presented in this section.
Going back to the algorithm \ref{eq:sym_strategy} we can apply to
\begin{align*}
\mathbf{v}_{t+1} = \frac{1}{2}(\text{BCH}(\tilde{\delta} \mathbf{v}^{L}_{t}, \mathbf{v}_{t}) + \text{BCH}(\mathbf{v}_{t},\tilde{\delta} \mathbf{v}^{R}_{t}))
\end{align*}
the parallel transport to get
\begin{align*}
\mathbf{v}_{t+1} &= \frac{1}{2}((\tilde{\delta} \mathbf{v}^{L}_{t})^{\parallel} + \mathbf{v}_{t} + \mathbf{v}_{t} + (\tilde{\delta} \mathbf{v}^{R}_{t})^{\parallel}) \\
&= 2\mathbf{v}_{t} + \frac{1}{2}((\tilde{\delta} \mathbf{v}^{L}_{t})^{\parallel} + (\tilde{\delta} \mathbf{v}^{R}_{t})^{\parallel})
\end{align*}
Applying the definition of parallel transport we get
\begin{align*}
(\tilde{\delta} \mathbf{v}^{L}_{t})^{\parallel} + (\tilde{\delta} \mathbf{v}^{R}_{t})^{\parallel} 
= 
\exp(-\frac{\mathbf{v}_{t}}{2}) \circ (\tilde{\delta} \mathbf{v}^{L}_{t} +\tilde{\delta} \mathbf{v}^{R}_{t} )\circ \exp(\frac{\mathbf{v}_{t}}{2})
\end{align*}
where 
\begin{align*}
\tilde{\delta} \mathbf{v}^{L}_{t} &=  \exp(\mathbf{v})\circ \exp(-\mathbf{v}_{t}) - e \\
\tilde{\delta} \mathbf{v}^{R}_{t} &=  \exp(-\mathbf{v}_{t})\circ \exp(\mathbf{v}) - e
\end{align*}
Then a new improvement of the algorithm \ref{eq:bossa_strategy}  is
\begin{equation}\label{eq:sym_parallel_strategy}
\begin{cases}
\mathbf{v}_0 = 0 \\
\mathbf{v}_{t} 
=  
2\mathbf{v}_{t-1} + \frac{1}{2}(\exp(-\frac{\mathbf{v}_{t-1}}{2}) 
\circ 
(\tilde{\delta} \mathbf{v}^{L}_{t-1} +\tilde{\delta} \mathbf{v}^{R}_{t-1} )\circ \exp(\frac{\mathbf{v}_{t-1}}{2}))
\end{cases}
\end{equation}
(This must be investigated!)